\chapter{Appendix A}
%The mean μAB,l and variance σ2AB,l could, in principle, be used to check for deviation from the H0 of independence at lag l, but in practice such a statistic would be corrupted by non-stationarities like (coupled) changes in the underlying firing rate (see Materials and methods, Figure 7, and Appendix on the importance of accounting for non-stationarity). Sliding window (Grün et al., 2002b) or bootstrap-based (Fujisawa et al., 2008; Pipa et al., 2008; Picado-Muiño et al., 2013) analyses have most commonly been used to deal with this issue, but come at the price of considerable data loss or computational burden. Here I suggest a simple remedy which corrects for non-stationarity locally by using the difference statistic #ABBA,l=#AB,l−#AB,−l (see Materials and methods, Figure 6B). The idea is that this way non-stationarities in firing rates would cancel out locally, on a comparatively fine time scale (≈lΔ), since they would affect #AB,l and #AB,−l alike (for assessment of synchronous spiking, I use #ABBA,0=#AB,0−#AB,l∗, with l∗=−2; see sect. ‘Choice of reference (correction) lag’ for the motivation of this particular choice and a more general discussion of the reference statistics chosen). The statistic Ql≡#ABBA,l2/σ̂ 2ABBA,l finally is approximately F-distributed and can be used for fast parametric assessment of the H0 (see Materials and methods and Figure 7; Figure 7—figure supplements 1 and 2, for derivation and empirical confirmation using non-stationary synthetic data).

%Having derived a fast, non-stationarity-corrected parametric test statistic for assessing the independence of pairs, I designed an agglomerative, heuristic clustering algorithm for fusing significant pairs into higher-order assemblies (see Figure 6—figure supplement 1 and Materials and methods for full derivation and pseudo-code). In essence, at each agglomeration step the algorithm treats each set of units fused in an earlier step just like a single unit with activation times defined through one of its member units. This allows for the same pair-wise test procedure on sets of units as defined for single units above, while at the same time effectively testing for higher-order dependencies based on the joint (set) distributions (see Materials and methods). Each pair is tested at all possible lags l∈{−lmax…lmax} (with lmax provided by the user), which is a reasonably fast process given the parametric evaluation introduced above. Should a pair of unit-sets prove significant at several lags l at any step, only the one associated with the minimum p-value is retained. The recursive set-fusing scheme stops if no more significant relationships among agglomerated sets and single units are detected. All subsets nested within larger sets are then discarded. This whole procedure is repeated for a set of user-provided bin widths Δ∈{Δmin…Δmax}. For each formed assembly, the width Δ∗ associated with the lowest p-value may then be defined as its characteristic temporal precision. All tests are performed at a user-specified, strictly Bonferroni-corrected α-level (always set to 0.05 here; see Materials and methods for details).