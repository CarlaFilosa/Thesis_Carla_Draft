\chapter{Cell assembly analysis}
\label{chap:AssemblyAnalysis}
In this chapter we present the assembly analysis conducted on the data set presented in \hyperref[chap:Dataset]{Chapter~\ref*{chap:Dataset}}, by using the cell assembly algorithm and the single units classification presented in \hyperref[chap:AssemblyMethod]{Chapter~\ref*{chap:AssemblyMethod}} and \hyperref[chap:UnitsAnalysis]{Chapter~\ref*{chap:UnitsAnalysis}} respectively.\\
The algorithm used is free to detect any spike pattern coordination at any time scale. Thus, applying the cell-assembly algorithm we were able to detect synchronous ($lag=0$) and asynchronous ($lag\neq 0$) cell assemblies at arbitrary time scale ($\Delta$). The time scales are explored running over different bin sizes of assembly-detection, and the lag, that is one outcome of the cell assembly algorithm, is the temporal distance in activation between units in assembly.\\
Since we are interested in cross areal interactions and directionality between Ventral Striatum (VS) and Ventral Tegmental Area (VTA), this work is focused on inter-regional assembly-pairs. Inter-regional assembly-pairs are assemblies formed by two neurons, of which one neuron is in Ventral Striatum and the other one is in Ventral Tegmental area. This is  an intuitive way to study the directionality between two regions: in fact at pairs-level, the lag in activation between units in assembly indicates the lag in activation between the two regions of concern. In our nomenclature a positive lag ($lag>0$) means that the VS unit is preceding in activation the VTA unit: VS is leading and VTA is following; whereas a negative lag ($lag<0$) means that the VTA unit is preceding the activation of the VS unit: we can say that VTA is leading and VS is following. 

\section{Cell types occurrence}
\label{sec:CellTypesOcc}
Using the units classification (\hyperref[chap:UnitsAnalysis]{Chapter~ \ref*{chap:UnitsAnalysis}}), inter-regional pairs were classified according to the cell-types in assembly. 
\begin{figure}[H]
    \centering
    \includegraphics[scale=0.35]{figures/PieRegions1.pdf}
    \includegraphics[scale=0.35]{figures/PieAsNotAs.pdf}
    \includegraphics[scale=0.35]{figures/PieAssembliesTot1.png}
    \caption{(Top) Occurrence of classified and not classified units in VS and VTA. (Middle) Occurrence of classified and not classified units of VS and VTA in inter-regional assembly-pairs. In VS, FSN occur in inter-regional pairs more often than SPN, even though more SPN than FSN were recorded.(Bottom) Pie charts of assemblies types. Not indicated percentage are $<1\%$. Missing pieces of cake indicates pairs that include not classified units. The four more represented inter-regional pairs, including only classified units, are pairs between fast spiking and gabaergic neurons ($20\%$), striatal projection neurons and dopaminergic neurons ($18\%$), fast spiking and dopaminergic units ($13\%$) and striatal projecton and gabaergic units ($9\%$). }
    \label{fig:PieAssembliesTot}
\end{figure}
Classified and not classified units of VS and VTA were recorded with the occurrence shown in fig. \ref{fig:PieAssembliesTot} (top), whereas they occur in assemblies as shown in fig. \ref{fig:PieAssembliesTot} (middle). Comparing the two pie charts related to the VS region, one observes fast spiking neurons (FSN) occur in inter-regional pairs more often than striatal projection neurons (SPN) (fig.\ref{fig:PieAssembliesTot} (middle-left)), even though recorded FSN are less than recorded SPN (fig. \ref{fig:PieAssembliesTot} (top-left)).\\We hypotesize that inter-regional pairs formation (or not) depends on the unit-types; to verify this hypothesis we conducted for each of the two regions a Pearson$'$s $\chi^2$ test. Of the classified unit types, only the more recurrent were tested namely striatal projecton neurons and fast spiking neurons in VS, and dopaminergic and gabaergic units in VTA. This last choice was made because cohlinergic interneuorns and glutamertegic units are poorly represented in the examinated data-set, and in the entire work we focus on the four more represented neuron types and pairs between those units.\\Each unit is part or not of an inter-regional pair, with this paradigm the contingencies tables \ref{tab:chi2_asnotasVS}, \ref{tab:chi2_asnotasVTA} were created, where the number (and the expected values in bracket) of specific cell types in inter-regional pairs or not in inter-regional pairs were reported, as well as the $\chi^2$ statistic value and the p-value. For each test the $\alpha$ significance level was fixed at $0.05$, unless differently specified.
\begin{table}[H]
    %\centering
\begin{tabular}{ |p{3cm}|p{3cm}|p{3cm}| }
 \hline
 \multicolumn{3}{|c|}{Pearson$'$s $\chi^2$ test VS unit type and inter-regional pair relationship} \\
 \hline
 & In pairs & Not in pairs\\
 \hline
 SPN & 153 (197.64) & 253 (208.36) \\
 \hline
 FSN & \textbf{197 (156.36)} & 116 (164.64)\\
 \hline
 \multicolumn{3}{|c|}{$\chi^2$ statistic  45.13}\\
 \multicolumn{3}{|c|}{p-value = $1.8\times10^{-11}$}\\
 \hline
 \multicolumn{3}{|c|}{$\chi^2$ statistic Yates correction 44.12}\\
 \multicolumn{3}{|c|}{p-value = $3.1\times10^{-11}$}\\
 \hline
\end{tabular}
\caption{Pearson$'$s $\chi^2$ contingency table with $\chi^2$ value and p-value. Unit types and inter-regional pairs formation are correlated in Ventral Striatum.}
\label{tab:chi2_asnotasVS}
\end{table}
\begin{table}[H]
    %\centering
\begin{tabular}{ |p{3cm}|p{3cm}|p{3cm}| }
 \hline
 \multicolumn{3}{|c|}{Pearson$'$s $\chi^2$ test VTA unit type and inter-regional pair relationship} \\
 \hline
 & In pairs & Not in pairs\\
 \hline
 DAN & 86 (90.604) & 31 (26.40) \\
 \hline
 GABA & 41 (36.40) & 6 (10.60)\\
 \hline
 \multicolumn{3}{|c|}{$\chi^2$ statistic  3.62}\\
 \multicolumn{3}{|c|}{p-value = 0.057}\\
 \hline
\end{tabular}
\caption{Pearson$'$s $\chi^2$ contigency table with $\chi^2$ value and p-value. Unit types and inter-regional pairs formation are not correlated in Ventral Tegmental Area.}
\label{tab:chi2_asnotasVTA}
\end{table}
A relationship between unit types and inter-regional pairs formation was found only in Ventral Striatum. In VS the $\chi^2$ statistic value is 45.13 (44.12 using Yates correction), that gives a p-value of $1.8\times10^{-11}$ ($3.1\times10^{-11}$), largely significant at $\alpha$ level. The result confirms our hypothesis that neuron-type in VS effects the formation of inter-regional assembly-pairs.\\The same test was conducted in VTA, with a resulting $\chi^2$ statistic of $3.62$ and a p-value of $0.057$, not significant at $\alpha = 0.05$ level. We conclude that in VTA different neuron-types have the same probability to form inter-regional pairs.\\
 We saw how often VS and VTA units occur in assembly, it remains to analyse if specific inter regional pair-types occur systematically more often than other. In pie-chart in fig. \ref{fig:PieAssembliesTot} (bottom) is shown the occurrence of assembly-types for the recorded units. Not indicated percentage are $<1\%$. Missing pieces of cake indicates pairs that include not classified units. Selecting only classified units, four assemblies types occurs more often than other, they are pairs formed by fast spiking and gabaergic neurons ($20\%$), striatal projection neurons and dopaminergic neurons ($18\%$), fast spiking and dopaminergic units ($13\%$) and striatal projecton and gabaergic units ($9\%$).\\
To see whether assembly types occur by chance or there is a relationship between the unit type activated in one region and the resulting assembly pairs, again Pearson's $\chi^2$ test were conducted. Specifically, given the pairs types occurrence, we hypothesize a preference by fast spiking neurons for gabaergic neurons (and/or vice-versa) and a preference by striatal projecton neurons for dopaminergic neurons (and/or vice-versa). The $\chi^2$ test were performed on the directional pairs ($lag\neq0$) and separately on $vs\rightarrow vta$ ($lag>0$) and $vs \leftarrow vta$ ($lag<0$). In both cases, the p-values of $\chi^2$ test were significant at the confidence level $\alpha = 0.05$, thus the $\chi^2$ test confirmed a dependence between the unit-type and the resulting inter-regional pair. In direction $vs\rightarrow vta$ $p=2\times10^{-4}$ ($p=4\times10^{-4}$ using Yates correction), in direction $vs \leftarrow vta$: $p=9\times10^{-3}$ ($p=0.017$ using Yates correction). In tables \ref{tab:chisquare_vsvta} and \ref{tab:chisquare_vtavs} are shown the contingency and the results of the $\chi^2$ tests for the two directionalities. In rows the activated cell types of the leading region, in columns the coupled selected cell types of the follower region. In the table-cells the number of pairs between the two cell-types and in brackets $()$ the expected values. Both in $vs\rightarrow vta$ and in $vs\leftarrow vta$ directionality the real values of couples $SPN+DAN$ and $FSN+GABA$ exceed the expected values.\\ 
\begin{table}[H]
    %\centering
\begin{tabular}{ |p{3cm}|p{3cm}|p{3cm}| }
 \hline
 \multicolumn{3}{|c|}{Pearson$'$s $\chi^2$ test ($vs \rightarrow vta$)} \\
 \hline
 & DAN pairs & GABA pairs\\
 \hline
 SPN & 76 (63.77) & 35 (47.23) \\
 \hline
 FSN & 32 (44.23) & 45 (32.77)\\
 \hline
 \multicolumn{3}{|c|}{$\chi^2$ statistic  13.47}\\
 \multicolumn{3}{|c|}{p-value = $2\times10^{-4}$}\\
 \hline
 \multicolumn{3}{|c|}{$\chi^2$ statistic Yates correction 12.39}\\
 \multicolumn{3}{|c|}{p-value = $4\times10^{-4}$}\\
 \hline
\end{tabular}
\caption{Pearson$'$s $\chi^{2}$ test contingency table. We test the dependency between the neuron type in VS and the neuron type in VTA with which the pair is formed, for pairs with specific directionality $vs \rightarrow vta$. The $\chi^2$ test show a dependency among variables, meaning that resulting pair depends on the neuron types involved.}
\label{tab:chisquare_vsvta}
\end{table}

\begin{table}[H]
\begin{tabular}{ |p{3cm}|p{3cm}|p{3cm}| }
 \hline
 \multicolumn{3}{|c|}{Pearson$'$s $\chi^2$ test ($vs \leftarrow vta$)} \\
 \hline
 & SPN pairs & FSN pairs\\
 \hline
 DAN & 18 (12.06) & 29 (34.94) \\
 \hline
 GABA & 11 (16.94) & 55 (49.06)\\
 \hline
 \multicolumn{3}{|c|}{$\chi^2$ statistic  6.73}\\
 \multicolumn{3}{|c|}{p-value = 0.009}\\
 \hline
 \multicolumn{3}{|c|}{$\chi^2$ statistic Yates correction 5.65}\\
 \multicolumn{3}{|c|}{p-value = 0.017}\\
 \hline
\end{tabular}
\caption{Pearson$'$s $\chi^{2}$ test contingency table. We test the dependency between the neuron type in VTA and the neuron type in VS with which the pair is formed, for pairs with specific directionality $vs \leftarrow vta$. The $\chi^2$ test show a dependency among variables, meaning that resulting pair depends on the neuron types involved.}
\label{tab:chisquare_vtavs}
\end{table}
\section{Cross-/intra- area interactions time scales}
\label{sec:TimeScales}
In the previous session we have seen that in Ventral Striatum the neuronal occurrence in assembly depends on the cell-types, and specifically pallidal units (FSN) occur more in assembly than striatal projection units. Furthermore we have seen that, in directional assembly, the combination among cell types is not casual, rather cell types prefer specific cell types to form inter-regional pairs.
With these analysis we exhaustively described the cell types occurrence in Ventral Striatum - Ventral Tegmental Area interactions.\\Time scales involved in the cross-area interactions remained to be examined and are argument of this chapter, together with a comparison with intra-area interaction time scales.\\
Detecting assembly at any time scale, the cell assembly-algorithm dissect the time scales involved in pairs-interactions. Instead to provide to the algorithm a specific bin-size for binning spike time series at which to detect spike patterns, a set of bin widths $\Delta \in \{\Delta_{min}...\Delta_{max}\}$ is provided, so that spike patterns can be detected at different bin-size, pairs are tested at all possible bin widths, then for each assembly, the width $\Delta^*$ associated with the lowest p-value may be defined as its characteristic temporal precision (see \hyperref[chap:AssemblyMethod]{Chapter~ \ref*{chap:AssemblyMethod}}).
In figure \ref{fig:BinDistr} is shown the temporal scale ($\Delta$) distribution of VS-VTA pair-interactions.
In figures \ref{fig:BinDistrVS} and \ref{fig:BinDistrVS} are shown VS-VS and VTA-VTA time-scale interactions distribution respectively.\\
\begin{figure}[H]
%\centering
\includegraphics[scale=0.5]{figures/VS_VTA_Short1.png}
\caption{Bin distribution for inter-regional pairs. VS-VTA pairs show a bimodal distribution, revealing two temporal scale involved in inter-regional activation patterns.}
\label{fig:BinDistr}
\end{figure}
\begin{figure}[H]
%\centering
\includegraphics[scale=0.5]{figures/VS_VS_S.png}
\caption{VS-VS pairs are more precise than VS-VTA pairs and the bin distribution presents a peak at 50 $ms$}
\label{fig:BinDistrVS}
\end{figure}
\begin{figure}[H]
%\centering
\includegraphics[scale=0.5]{figures/VTA_VTA_S.png}
\caption{VTA-VTA temporal scale distribution does not present any peak.}
\label{fig:BinDistrVTA}
\end{figure}
A comparison among inter-regional pairs (VS-VTA pairs) and intra-regional pairs (VS-VS pairs and VTA-VTA pairs) show interesting differences, that we are going to analyse.\\
While we observed assemblies of temporal precision at the scale of few tens of milliseconds only within either VS or VTA, assemblies of lower temporal precision were detected across VS-VTA units. Inter-regional VS-VTA interactions have a bimodal time-scales distribution with peaks around 80 $ms$ and one 1.6 $sec$, revealing two time scales involved in VS-VTA interaction, the first, more precise, ranged from 10 $ms$ to 250 $ms$, and the second including broader bin sizes; from which we consequently argue a complex interaction circuit effect, reflecting in inter-regional pairs.\\Bimodality is a characteristic specific of VS-VTA interactions, not present in VTA-VTA, or VS-VS interactions: intra-regional VTA-VTA pairs do not present any peak in time scales distribution, whereas intra-regional VS-VS bin size distribution is peaked around 50 $ms$.  
\subsection{SPN-FSN time scales interactions *}
\label{sec:SPN-FSN_Bin}
Fast spiking neurons population has broad firing rate, and according to the firing rate, sub-populations of the neurons classified as fast spiking neurons in first place can show different characteristic in terms of time-scales and length of interactions, or/ and feature coding ({\color{red}ask for paper to cite}).\\From the distribution of mean firing rate of the recorded FSN was possible to define two sub-populations, those last do not present different characteristic when their units are coupled in assembly with a VTA neuron, however it worth to mention them in relation to their time-scales interactions with SPN. We report in figure \ref{fig:FSNsFireHisto} the histogram of FSNs mean firing rate, from which is possible to distinguish two sub-populations, that we will call FSNs low and FSNs high: the first characterized to have mean firing rate below $45 Hz$ and the latter has mean firing rate equal or above $45 Hz$.\\
\begin{figure}
    \centering
    \includegraphics[scale=0.6]{figures/FSNFiringRateLightDark.pdf}
    \caption{Histogram of FSNs mean firing rate. We can distinguish two populations of FSNs: FSN low-firing-rate population (FSN-low), that are light grey in the graph, characterized by having a firing rate below 45 $Hz$, and FSN high-firing-rate population (FSN-high), in dark grey, characterized by having a firing rate from 45 $Hz$ upwards.}
    \label{fig:FSNsFireHisto}
\end{figure}
\begin{figure}
    \centering
    \includegraphics[scale=0.5]{figures/SPN_FSNlow1.pdf}
    \caption{SPN-FSN-low temporal scale distribution is peaked at 50 $ms$. A good portion of SPN-FSN-low pairs is detected also at 80 $ms$ and 120 $ms$.}
    \label{fig:SPN_FSNlowBin}
\end{figure}
\begin{figure}
    \centering
    \includegraphics[scale=0.5]{figures/SPN_FSNhigh1.pdf}
    \caption{SPN-FSN-high pairs are almost exclusively detected at very precise time scales, namely from 10 $ms$ to 50 $ms$.}
    \label{fig:SPN_FSNhighBin}
\end{figure}
In VS we noticed differences between SPN-FSN-low pairs and SPN-FSN-high bin size distributions. SPN-FSN-low bin distribution pair is peaked at 50 $ms$, and another good portion of those pairs is detected at the two next bin sizes after the peak, 80 $ms$ and 120 $ms$ (fig. \ref{fig:SPN_FSNlowBin}); whereas SPN-FSN-high pairs are essentially only detected at more precise temporal scale (fig.\ref{fig:SPN_FSNhighBin}).\\We conclude that those two pair-types give a specific contribution to the global VS-VS temporal scale distribution. Once more we have a confirm of the variety of time scales involved in intra- or cross- area interactions in the studied regions, emphasizing the complexity of the interaction circuit.
\section{Directionality} 
\label{sec:Directionality}
The conclusion made in \hyperref[sec:TimeScales]{Section~\ref*{sec:TimeScales}} on inter-regional interactions time scales led us to consider separately precise ($\Delta \in [0.01,0.25] ms$) and broad ($\Delta \in [0.25,0.6] ms$) VS-VTA pairs. We divided then these two assembly populations in the further study of directionality.\\We recall that one of the output of the cell-assembly algorithm is the lag in activation between units in assembly, and when we study inter-regional pairs the lag value tells us the distance in activation between the two region and the sign of the lag indicates the direction of the activation, namely which region became activated first and which one follows.\\
In fig.\ref{fig:LagInSecAll} we show the lag distribution for detected inter-regional pairs in the two time scales. As indicated in the plot, positive lag means VS is functionally leading VTA activation and negative lag indicates the opposite direction.\\Interestingly lag distributions of precise and broad pairs are asymmetric, indicating that the VS activation leads the activation of the VTA neuron. The two distributions show however remarkable differences: the lag distribution of broader pairs is fat-long tailed, that means a good portion of pairs detected with long activation lag($lag > 1 sec$), whereas precise time scales lag distribution has thin tails: almost all pairs detected in precise time scale have short lag ($|lag| < 1 sec$), and a good portion of pairs is detected within a lag value of 0.5 $sec$.\\
We focused the study on the more precise temporal scale, in such a way that temporal scale interactions were separated to typical task-related time scales, such as the length of the odor duration, which cover typically an interval from 1.0 $sec$ to 1.5 $sec$.\\
We observed that, the directional assemblies are composed of striatal projection neurons leading dopaminergic neurons (SPN-DAN pairs), all the other pair-types do not show a clear preferred direction. Furthermore, inter unit activation lags of assemblies containing pallidal neurons (FSN) were shorter than those containing striatal projection neurons (SPN), compatibly with assumed connectivity. In fig.\ref{fig:LagInSec4typo} is shown the lag distribution for the four principal pair-types.\\
As control for dopaminergic neurons classification, VTA dopaminergic were also laser tagged, in fig.\ref{fig:LagInSecLaser} is shown the lag distribution of laser tagged dopaminergic units coupled with SPN and FSN. We can observe the similarity of the lag distribution of pairs with laser tagged dopaminergic units and the lag distribution of the pairs formed by classified dopaminergic units, SPN-laser DAN pairs show directionality in direction $VS\rightarrow VTA$, whereas FSN-laser DAN are not directional, as we expected from the results obtained using classified units, that confirms the validity of the adopted classification.\\
\begin{figure}[H]
\centering
\includegraphics[scale=0.65]{figures/LagGeneral1.pdf}
\caption{Lag distribution for VS-VTA pairs in seconds. In green the synchronous pairs. On the left, lag distribution for pairs detected in preciser time scales. Slight distribution asymmetry indicates directionality in the direction of $lag > 0$, meaning a predominance of pairs in which VS leads VTA. On the right, the lag distribution for pairs detected in the broader time scale, it presented as an asymmetric fat-tailed distribution.}
\label{fig:LagInSecAll}
\end{figure}
\begin{figure}[H]
\centering
\includegraphics[scale=0.5]{figures/LagSec4Typo3VS.png}
\caption{Lag distribution in seconds of four more represented pair-types in precise time scale. Only SPN-DAN pairs show a preferred direction, which is $VS\rightarrow VTA$.}
\label{fig:LagInSec4typo}
\end{figure}
\begin{figure}[H]
\centering
\includegraphics[scale=0.5]{figures/LagSecLaser3VS.png}
%\includegraphics[scale=0.4]{figures/OnlyLaserOriz.png}
\caption{Lag distribution in seconds of laser tagged dopaminergic units in pair with SPN and FSN units, in precise time scale. Lag distributions are similar to the lag distributions of classified DAN in pair with SPN and FSN, confirming the directionality expressed by SPN- DAN pairs and the validity of the classification adopted in VTA.}
\label{fig:LagInSecLaser}
\end{figure}
Time precision and lag analysis reveal time scale segregation, that reveals in turn different assembly-activation patterns. An example is provided in fig.\ref{fig:AsActBinLag}, we show the heat map of inter-regional pairs activity averaged across trials in one of the paradigms of the experiment. Time scale segregation means segregation in activation patterns. Pairs with directionality $VS \rightarrow VTA$ and $\Delta \le 0.25$ became activated early during the odour presentation, a good portion of them show a second activation after the odour presentation.\\The early activation at the stimulus presentation is the type of activation we expect when the animal predicts the reward, as discussed in \hyperref[sec:StateArt]{Section~\ref*{sec:StateArt}}, moreover we have shown in fig.\ref{fig:LagInSec4typo} as the SPN-DAN pairs are directional, thus we speculate that SPN-DAN are good candidate for reward prediction (RP) coding; the investigation of this hypothesis will be argument of the next chapters.\\
\begin{figure}
    \centering
    \includegraphics[scale=0.45]{figures/AsActPerBinLag1.png}
    \caption{Assembly-activation patterns given time bins and lags. In a.) heat map of inter-regional pairs in one the experimental paradigms, b,c,d. pairs of a. selected for bin size ($\Delta$) and lag: $\Delta < 0.25 s$ and $lag > 0$ (b.), $\Delta > 0.25 s$ and $lag > 0$ (c.), $lag < 0$ (d.)}
    \label{fig:AsActBinLag}
\end{figure}
To visualize better the stimulus activation of directional pairs, we give an example in fig.\ref{fig:directional_assembly}. The pair in example was detected with a bin width of $\Delta = 0.12$, and the lag in activation between the units was positive with value 0.36 $sec$. On top is shown the mean across trials of the pair's activity for rewarded (grey line) and unrewarded odor (purple line) in the original phase. The pair in example discriminates between paired and unpaired odour and specifically code for rewarded odor. On bottom raster plot and mean firing rate of units in assembly. x-axis origin corresponds to the odour onset, while the red line marks the end of the stimulus duration.
Looking at the activity of two neurons the directionality exemplified by the pair is evident, in fact we see in first place the VS unit becoming active and, after a temporal delay, the activation of VTA unit following. Raster plots show the stability of the activity across trials.  
\begin{figure}
    \centering
    \includegraphics[scale=0.6]{figures/DirectionalAsEx1.pdf}
    \caption{Directional assembly. On top mean across trails and standard error of the activity of the pair for rewarded (grey) and unrewarded odour (purple) in the original phase. On bottom raster plot and firing rate (mean and standard deviation) of units in assembly. x-axis origin corresponds to the odor onset, while the red line marks the end of the stimulus duration. The examined assembly has a positive lag, that means VS preceding in activity VTA, from neuronal activity we can see in fact the VS unit activate before than the VTA unit.}
    \label{fig:directional_assembly}
\end{figure}
\section{Conclusion}
From the inter-regional time scales distribution (bin size $\Delta$) we deduced that VS-VTA interactions are two time scales interactions. Looking at the assembly activity patterns we observed a dissection between the more precise time scale and the broader time scale inter-regional pairs. 
With the lags between units activation we have distinguished three types of assemblies: synchronous, positive directional and negative directional assemblies. From lag distributions one deduces that precise and broad time scale pairs show directionality in the direction of VS leading VTA.
Directional assemblies are formed by Striatal Projection Neurons and Dopaminergic Neurons.
Do directional assemblies show a different coding features with respect to other assemblies types?
To address this question we investigate the pair-types task related pattern.
  
 \section{Pair-types task related response}
 %%%%%%%% Commento forse da buttare
 %First difference between original and reversal phase is number of assemblies responding during the post-stimulus interval, this number decrease for all the typologies involved during hit trials (as shown in fig. (\ref{fig:histo_taskrel})), in according to the intuition, in reversal phase in fact, when the animals became more expert, neuronal response tend to be shifted closer to the reward delivery. This effect can be seen in fig. (\ref{fig:NeusInAsse}) where activity and raster plots of two units in a pair are shown. Looking at the raster plots, a shift in correspondence to the phase-switch is evident in both units.
 %\begin{figure}
  %   \centering
  %   \includegraphics[scale=0.3]{figures/Original_Hit_N.jpg}
    % \includegraphics[scale=0.3]{figures/Reversal_Hit_N.jpg}
     %\caption{{\color{red}TO MODIFY!!!! BUT THIS IS THE PLOT}}
     %\label{fig:histo_taskrel}
 %\end{figure}
%\begin{figure}
  %  \centering
   % \includegraphics[scale=0.6]{figures/SingleNeus1_15Lastrev1Pru_An_4.jpg}
    %\caption{Shift in time of neuronal activity of two units in assembly. }
    %\label{fig:NeusInAsse}
%\end{figure}

%\section{Discussion}
%\section{Combination of single neuron and assemblies analysis}
%\subsection{Directionality using classification}
%\subsection{Significant task related response for typology}
%%%%%%% Fine Commento forse da buttare


%%%%% Commento Utile
%To better study assemblies activation patterns, first the task relevant moments of the experiment were selected. From the mean task related activity patterns we expected to see differences among assemblies types in two experimental chapters (original and reversal). To better visualize the task related activation patterns via heat plots, hit trials (rewarded odor, mouse went for reward), correct rejection trials (unrewarded odor, mouse sat quiet), false alarm trials (unrewarded odor, mouse went for reward), were kept separated; however this separation among trials types was released in further analysis, without affecting results.
%The assemblies were pruned according their significant task related activity, that was tested with Friedman's test and a non parametric version of the repeated measures Anova. We preferred to use non-parametric tests to be free from the assumption of gaussianity of the observations. Results of the two tests were consistent each other. The two relevant events of the task were the odor onset and the reward delivery, then we choose whether the assemblies showed a significant activity in three windows: Stimulus [0s, 0.5s], Pre-Reward [-0.5s, 0s], Reward [0s, 05s], the baseline was chosen in the interval [-1s, -0.5s] from the odor onset. Post-hoc analysis were performed using the Bonferroni's criterion {\color{red}check whether the criterion was Bonferroni of some other}. Almost $80\%$ of the VS-VTA assemblies showed a task related activity significant different from the baseline or from another of the windows considered. Of the significant assemblies $\%$ were composed by MSN-Type I units, $\%$ by FSI low-Type I, $\%$ by FSI high-Type I, $\%$ MSN-Type II, $\%$ by FSI low-Type II, $\%$ by FSI high-Type II, the other possible units combinations constitutes a minority and all toghether were the $\%$ {\color{red} Insert numbers of percentage}.

%\section{Conclusion}
