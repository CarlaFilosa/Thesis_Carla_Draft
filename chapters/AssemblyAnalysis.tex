\chapter{Cell assembly analysis}
\label{chap:AssemblyAnalysis}
In this chapter we present the assembly analysis conducted on the data set introduced in \hyperref[sec:Dataset]{Section~\ref*{sec:Dataset}}, using the cell assembly detection algorithm and the single units classification presented in \hyperref[chap:AssemblyMethod]{Chapter~\ref*{chap:AssemblyMethod}} and \hyperref[chap:UnitsAnalysis]{Chapter~\ref*{chap:UnitsAnalysis}} respectively.\\
The cell assembly detection algorithm is designed to detect any coordinated spike trains patterns at any time scale. The time scales are explored running over different bin-widths. Further we examined the lag distribution. Lags describe the temporal distance in activation between units in assembly. Applying the cell-assembly algorithm we detected synchronous ($lag=0$) and asynchronous ($lag\neq0$) cell assemblies at arbitrary time scale ($\Delta$).\\
We were interested in cross areal interactions and directionality between ventral striatum (VS) and ventral Tegmental area (VTA). interregional assembly-pairs are assemblies formed by two neurons, of which one neuron is located in  the VS and the other one in VTA. At the pair-level, the lag in activation between units in assembly indicates the lag in activation between the two regions. A positive lag ($lag>0$) indicates that the VS unit is preceding in its activation the VTA unit, or in other words: VS is leading and VTA is following. A negative lag ($lag<0$) means that the VTA unit is preceding the activation of the VS unit, hence VTA is leading and VS is following.
\section{Cell types occurrence}
\label{sec:AsseTypes}
Based on the unit classification (\hyperref[chap:UnitsAnalysis]{Chapter~ \ref*{chap:UnitsAnalysis}}), interregional assembly were classified according to their underlying cell-types (figure \ref{fig:PieAssembliesTot} (B.)).
Comparing the two pie charts related to the VS region, one observes fast spiking neurons (FSN) occur in interregional pairs more often than striatal projection neurons (SPN) (figure \ref{fig:PieAssembliesTot} (B.-left)), even though recorded FSN are less than recorded SPN (figure \ref{fig:PieAssembliesTot} (A.-left)).
\label{sec:CellTypesOcc}
\begin{figure}[H]
    \centering
    \includegraphics[scale=0.35]{figures/PieRegions1.pdf}
    \includegraphics[scale=0.35]{figures/PieAsNotAs.pdf}
    \includegraphics[scale=0.35]{figures/PieAssembliesTot1.png}
    \caption{(A.) Occurrence of classified and not classified units in VS and VTA. (B.) Occurrence of classified and not classified units of VS and VTA in interregional assembly-pairs. In VS, FSN occur in interregional pairs more often than SPN, even though more SPN than FSN were recorded.(C.) Pie charts of assemblies types. Not indicated percentage are $<1\%$. Missing pieces of cake indicates pairs that include not classified units. The four more represented interregional pairs, including only classified units, are pairs between fast spiking and gabaergic neurons ($20\%$), striatal projection neurons and dopamine neurons ($18\%$), fast spiking and dopamine units ($13\%$) and striatal projection and gabaergic units ($9\%$). }
    \label{fig:PieAssembliesTot}
\end{figure}
We hypothesize specific cell-types have a higher tendency to aggregate into cell assemblies; to verify this hypothesis we conducted for each of the two regions a Pearson's $\chi^2$ test. Of the classified unit types, only the ones detected at sufficient frequencies could tested for reasons of statistical power: namely putative striatal projection neurons (pSPN) and fast spiking neurons (FSN) in VS, and dopamine and gabaergic units in VTA. Only few striatal cholinergic interneurons and VTA glutamatergic units were detected in the examined data-set, and therefore not amenable to statistical analysis. We therefore focused on the four most prominent neuron types and the cell assembly pairs formed between those units.\\Whether the recorded unit may or may not be part of an interregional pair is described  in the contingencies tables \ref{tab:chi2_asnotasVS}, \ref{tab:chi2_asnotasVTA}. In the contingencies tables, the number (compared to the expected values indicated in parentheses) of specific cell types in interregional pairs were reported with $\chi^2$ statistic p-values. For each test the $\alpha$ significance level was fixed at $0.05$, unless otherwise specified.\\
\begin{table}[h!]
    %\centering
\begin{tabular}{ |p{3cm}|p{3cm}|p{3cm}| }
 \hline
 \multicolumn{3}{|c|}{Pearson$'$s $\chi^2$ test VS unit type and interregional pair relationship} \\
 \hline
 & In pairs & Not in pairs\\
 \hline
 SPN & 153 (197.64) & 253 (208.36) \\
 \hline
 FSN & \textbf{197 (156.36)} & 116 (164.64)\\
 \hline
 \multicolumn{3}{|c|}{$\chi^2$ statistic  45.13}\\
 \multicolumn{3}{|c|}{p-value = $1.8\times10^{-11}$}\\
 \hline
 \multicolumn{3}{|c|}{$\chi^2$ statistic Yates correction 44.12}\\
 \multicolumn{3}{|c|}{p-value = $3.1\times10^{-11}$}\\
 \hline
\end{tabular}
\caption{Pearson's $\chi^2$ contingency table with $\chi^2$ value and p-value. Unit-types and interregional pairs formation are correlated in ventral striatum.}
\label{tab:chi2_asnotasVS}
\end{table}\\
\begin{table}[h!]
    %\centering
\begin{tabular}{ |p{3cm}|p{3cm}|p{3cm}| }
 \hline
 \multicolumn{3}{|c|}{Pearson's $\chi^2$ test VTA unit-type and interregional pair relationship} \\
 \hline
 & In pairs & Not in pairs\\
 \hline
 DAN & 86 (90.604) & 31 (26.40) \\
 \hline
 GABA & 41 (36.40) & 6 (10.60)\\
 \hline
 \multicolumn{3}{|c|}{$\chi^2$ statistic  3.62}\\
 \multicolumn{3}{|c|}{p-value = 0.057}\\
 \hline
\end{tabular}
\caption{Pearson's $\chi^2$ contingency table with $\chi^2$ value and p-value. Unit-types and interregional pairs formation are not correlated in ventral Tegmental area.}
\label{tab:chi2_asnotasVTA}
\end{table}
A relationship between unit types and interregional pairs formation was found only in VS. In VS the $\chi^2$ statistic value is 45.13 (44.12 using Yates correction), that gives a p-value of $1.8\times10^{-11}$ ($3.1\times10^{-11}$). The result confirms that in VS the tendency of being agglomerate in interregional pairs depends on the specific cell-type.\\A similar test was conducted in VTA, with a resulting $\chi^2$ statistic of $3.62$ and a p-value of $0.057$, not significant at $\alpha = 0.05$ level. We therefore conclude that in VTA different cell-types have the same probability to agglomerate in interregional pairs.\\
 We have shown above how often VS and VTA units occur in assemblies. We next analyzed if specific inter regional pair-types occur systematically more often than others. The occurrence of assembly-types for the recorded units is shown in the pie-chart of figure  \ref{fig:PieAssembliesTot} (bottom). Piece of chart without displayed percentages refer to pairs occurring with a rate below $< 1\%$.  Missing pieces of cake indicates pairs that include not classified units. Selecting only classified units, four assemblies types occurred more often than other, they were pairs formed by fast spiking and gabaergic neurons (20$\%$), striatal projection neurons and dopamine neurons (18$\%$), fast spiking and dopamine units (13$\%$) and striatal projection and gabaergic units (9$\%$).\\
To see whether assembly types occur by chance and whether there is a relationship between the unit type activated in one region and the resulting assembly pairs, again a Pearson's $\chi^2$ test was conducted. Specifically, given the relative frequency of certain types of assemblies, we hypothesize a preference for fast spiking neurons with gabaergic neurons (and/or vice-versa) and a preference for striatal projection neurons with dopamine neurons (and/or vice-versa). The $\chi^2$ test were performed on the directional pairs ($lag\neq0$) and separately on $VS\rightarrow VTA$ ($lag>0$) and $VS\leftarrow VTA$ ($lag<0$). In both cases, the p-values of $\chi^2$ test were significant at the confidence level $\alpha = 0.05$, hence the $\chi^2$ test confirmed a dependence between the cell-type and the resulting interregional assembly pair type. In direction $VS\rightarrow VTA$ the p-value was $2\times10^{-4}$ ($p=4\times10^{-4}$ using Yates correction), in direction $VS\leftarrow VTA$: $p=9\times10^{-3}$ ($p=0.017$ using Yates correction). The contingency and the results of the $\chi^2$ tests are shown for the two directionalities in tables \ref{tab:chisquare_vsvta} and \ref{tab:chisquare_vtavs}. The activated cell types of the leading region are indicated in the rows, the coupled selected cell types of the follower region in the columns. In the table-cells the number of pairs between the two cell-types and in brackets the expected values. Both in $VS\rightarrow VTA$ and in $VS\leftarrow VTA$ directionality the real values of couples $SPN+DAN$ and $FSN+GABA$ exceed the expected values. In both directionality specific pair-types have an higher tendency to occur.\\ 
\begin{table}[H]
    %\centering
\begin{tabular}{ |p{3cm}|p{3cm}|p{3cm}| }
 \hline
 \multicolumn{3}{|c|}{Pearson's $\chi^2$ test ($VS \rightarrow VTA$)} \\
 \hline
 & DAN pairs & GABA pairs\\
 \hline
 SPN & 76 (63.77) & 35 (47.23) \\
 \hline
 FSN & 32 (44.23) & 45 (32.77)\\
 \hline
 \multicolumn{3}{|c|}{$\chi^2$ statistic  13.47}\\
 \multicolumn{3}{|c|}{p-value = $2\times10^{-4}$}\\
 \hline
 \multicolumn{3}{|c|}{$\chi^2$ statistic Yates correction 12.39}\\
 \multicolumn{3}{|c|}{p-value = $4\times10^{-4}$}\\
 \hline
\end{tabular}
\caption{Pearson's $\chi^{2}$ test contingency table. We test the dependency between the neuron type in VS and the neuron type in VTA with which the pair is formed, for pairs with specific directionality $VS \rightarrow VTA$. The $\chi^2$ test show a dependency among variables, meaning that specific pairs have higher tendency to agglomerate in assembly.}
\label{tab:chisquare_vsvta}
\end{table}
\begin{table}[H]
\begin{tabular}{ |p{3cm}|p{3cm}|p{3cm}| }
 \hline
 \multicolumn{3}{|c|}{Pearson$'$s $\chi^2$ test ($VS \leftarrow VTA$)} \\
 \hline
 & SPN pairs & FSN pairs\\
 \hline
 DAN & 18 (12.06) & 29 (34.94) \\
 \hline
 GABA & 11 (16.94) & 55 (49.06)\\
 \hline
 \multicolumn{3}{|c|}{$\chi^2$ statistic  6.73}\\
 \multicolumn{3}{|c|}{p-value = 0.009}\\
 \hline
 \multicolumn{3}{|c|}{$\chi^2$ statistic Yates correction 5.65}\\
 \multicolumn{3}{|c|}{p-value = 0.017}\\
 \hline
\end{tabular}
\caption{Pearson$'$s $\chi^{2}$ test contingency table. We test the dependency between the neuron type in VTA and the neuron type in VS with which the pair is formed, for pairs with specific directionality $VS \leftarrow VTA$. The $\chi^2$ test show a dependency among variables, meaning that specific pairs have higher tendency to agglomerate in assembly.}
\label{tab:chisquare_vtavs}
\end{table}
\section{Inter-/intra- regional pair time scales}
\label{sec:TimeScales}
In the previous session we have seen that in ventral striatum the neuronal occurrence in assembly depends on the cell-types, and specifically pallidal units (FSN) occur more in assembly than striatal projection units (SPN). Furthermore we have seen that, in directional assembly, the combination among cell types is non-random, rather cell-types prefer specific cell-types to which agglomerate in interregional pairs.
With these analysis we exhaustively described the cell types occurrence in ventral striatum-ventral Tegmental area interactions.\\Time scales involved in the cross-area interactions remained to be examined and are argument of this chapter, together with a comparison with intra-area interaction time scales.\\
Detecting assemblies at any time scale, the detection algorithm dissect the time scales involved in pairs-interactions. A set of bin widths $\Delta \in \{\Delta_{min}...\Delta_{max}\}$ is provided as input, so that spike patterns can be detected at different bin-size, pairs are tested at all possible bin widths, then for each assembly, the width $\Delta^*$ associated with the lowest p-value may be defined as its characteristic temporal precision (\cite{RussoDurstewitz}, see \hyperref[chap:AssemblyMethod]{Chapter~ \ref*{chap:AssemblyMethod}}).
In figure \ref{fig:BinDistr} is shown the temporal scale ($\Delta$) distribution of VS-VTA pair-interactions.\\
In figures \ref{fig:BinDistrVS} and \ref{fig:BinDistrVS} are shown VS-VS and VTA-VTA time-scale interactions distribution respectively.\\
\begin{figure}[h!]
%\centering
\includegraphics[scale=0.46]{figures/VS_VTA_Short1.png}
\caption{Bin distribution for interregional pairs. VS-VTA pairs show a bimodal distribution, revealing two temporal scale involved in interregional activation patterns.}
\label{fig:BinDistr}
\end{figure}
\begin{figure}[h!]
%\centering
\includegraphics[scale=0.46]{figures/VS_VS_S.png}
\caption{VS-VS pairs are more precise than VS-VTA pairs and the bin distribution presents a peak at 50 $ms$}
\label{fig:BinDistrVS}
\end{figure}
\begin{figure}[h!]
%\centering
\includegraphics[scale=0.46]{figures/VTA_VTA_S.png}
\caption{VTA-VTA temporal scale distribution does not present any peak.}
\label{fig:BinDistrVTA}
\end{figure}
A comparison among interregional pairs (VS-VTA pairs) and intraregional pairs (VS-VS pairs and VTA-VTA pairs) show interesting differences, that we are going to analyse.\\
While we observed assemblies of temporal precision at the scale of few tens of milliseconds only within either VS or VTA, assemblies of lower temporal precision were detected across VS-VTA units. interregional VS-VTA interactions have a bimodal time-scales distribution with two peaks, one around 80 $ms$ and one at 1.6 $sec$, revealing two time scales involved in VS-VTA interaction. The first, more precise time scale, ranges from 10 $ms$ to 250 $ms$, and the second includs broader bin sizes. From the bimodality we consequently argue a complex interaction circuit effect, reflecting in interregional pairs.\\Bimodality is a characteristic specific of VS-VTA interactions, not present in VTA-VTA, or VS-VS interactions: intraregional VTA-VTA pairs do not present any peak in time scales distribution, whereas intraregional VS-VS bin size distribution is peaked around 50 $ms$.  
\subsection{SPN-FSN time scales interactions *}
\label{sec:SPN-FSN_Bin}
Fast spiking neurons population has broad firing rate, and according to the firing rate, sub-populations of the neurons classified as fast spiking neurons in first place can show different characteristic in terms of time-scales and length of interactions, or/and feature coding ({\color{red}ask for paper to cite}).\\From the distribution of mean firing rate of the recorded FSN was possible to define two sub-populations, those last do not present different characteristic when their units are coupled in assembly with a VTA neuron, however it is worth to mention them in relation to their time-scales interactions in intraregional pairs with SPN. We report in figure \ref{fig:FSNsFireHisto} the histogram of FSNs mean firing rate, from which is possible to distinguish two sub-populations, that we will call FSNs low and FSNs high: the first characterized to have mean firing rate below $45 Hz$ and the latter has mean firing rate equal or above $45 Hz$.\\
\begin{figure}
    \centering
    \includegraphics[scale=0.6]{figures/FSNFiringRateLightDark.pdf}
    \caption{Histogram of FSNs mean firing rate. We can distinguish two populations of FSNs: FSN low-firing-rate population (FSN-low), that are light grey in the graph, characterized by having a firing rate below 45 $Hz$, and FSN high-firing-rate population (FSN-high), in dark grey, characterized by having a firing rate from 45 $Hz$ upwards.}
    \label{fig:FSNsFireHisto}
\end{figure}
\begin{figure}
    \centering
    \includegraphics[scale=0.5]{figures/SPN_FSNlow1.pdf}
    \caption{SPN-FSN-low temporal scale distribution is peaked at 50 $ms$. A good portion of SPN-FSN-low pairs is detected also at 80 $ms$ and 120 $ms$.}
    \label{fig:SPN_FSNlowBin}
\end{figure}
\begin{figure}
    \centering
    \includegraphics[scale=0.5]{figures/SPN_FSNhigh1.pdf}
    \caption{SPN-FSN-high pairs are almost exclusively detected at very precise time scales, namely from 10 $ms$ to 50 $ms$.}
    \label{fig:SPN_FSNhighBin}
\end{figure}
In VS we noticed differences between SPN-FSN-low pairs and SPN-FSN-high bin size distributions. SPN-FSN-low bin distribution pair is peaked at 50 $ms$, and another good portion of those pairs is detected at the two next bin sizes after the peak, 80 $ms$ and 120 $ms$ (figure \ref{fig:SPN_FSNlowBin}); whereas SPN-FSN-high pairs are essentially only detected at more precise temporal scale (figure \ref{fig:SPN_FSNhighBin}).\\We conclude that those two pair-types give a specific contribution to the global VS-VS temporal scale distribution. The variety of time scales involved in intra- or cross- area interactions in the studied regions emphasizes the complexity of the interaction circuit.
\section{Directionality} 
\label{sec:Directionality}
We had found in \hyperref[sec:TimeScales]{Section~\ref*{sec:TimeScales}} that interregional interactions have two characteristic time scales, which led us to consider more precise ($\Delta \in [0.01,0.25] ms$) and broader ($\Delta \in [0.25,0.6] ms$) time scales separately in the further study on directionality of VS-VTA assembly-pairs.\\We recall that one of the output of the cell-assembly algorithm (CAD) is the inter-units activation lag of the assembly. When we restrict the investigation to interregional pairs, the lag value tells us the distance in activation between the two region while the sign of the lag indicates the direction of the activation, namely which region became activated first and which one follows.\\In figure \ref{fig:LagInSecAll} we show the lag distribution for detected interregional assembly-pairs in the two characteristic time scales of interaction. As indicated in the plot, a positive lag means that VS is functionally leading the VTA activation and a negative lag indicates the opposite direction.\\Interestingly lag distributions of precise and broad pairs are asymmetric, indicating that preferentially the VS activation leads the activation of the VTA. The two lag distributions show however remarkable differences: the lag distribution of broader pairs is fat-long tailed, indeed, a good portion of assembly-pairs detected has long activation lag ($lag > 1 sec$); more precise pairs lag distribution has instead thin tails: almost all pairs detected in precise time scale have short lag ($|lag| < 1 sec$), and a good portion of pairs is detected within a lag value of 0.5 $sec$.\\
We focused the study on the more precise temporal scale, first because in such a way that temporal scale interactions were separated to typical task-related time scales, as e.g. the length of the odor duration, which cover typically an interval from 1.0 $sec$ to 1.5 $sec$, secondly, but not less importantly, because the prediction error functions are expressed by dopamine circuit in short time scales (within 1 $sec$ from stimulus onset).\\ We used the cell-types classification presented in\hyperref[chap:UnitsAnalysis]{~Section \ref*{chap:UnitsAnalysis}} to divide the interregional assembly-pairs in assembly-pair types, according to their underlying cell-types (see \hyperref[sec:AsseTypes]{~Section \ref*{sec:AsseTypes}}). Doing so, we could divide the global lag distribution in lag distributions specific of those assembly-pairs types.\\
In figure \ref{fig:LagInSec4typo} is shown the lag distribution for the four principal assembly-pair types. The green bar indicates the synchronous assembly-pairs, whereas directional assembly-pairs are colored in violet.\\
We observed that the directional assemblies are composed of striatal projection neurons leading dopamine neurons (SPN-DAN pairs), all the other pair-types do not show a clear preferred directionality. Furthermore, inter-unit activation lags of assemblies containing pallidal neurons (FSN) are shorter than those containing striatal projection neurons (SPN), compatibly with assumed connectivity.\\
VTA dopamine neurons were laser tagged in first place, we can use the laser tagged units as control to validate the dopamine neurons classification; in figure \ref{fig:LagInSecLaser} is shown the lag distribution of laser tagged dopamine units coupled with SPN and FSN. Is clear the similarity between the lag distribution of pairs containing laser tagged dopamine units and the lag distribution of the pairs containing classified dopamine units. Indeed, SPN-laser DAN pairs show directionality in direction $VS\rightarrow VTA$, whereas FSN-laser DAN are not directional, as we expected from the results obtained using classified units, which confirms the validity of the adopted classification.\\
\begin{figure}[H]
\centering
\includegraphics[scale=0.58]{figures/LagGeneral1.pdf}
\caption{Lag distribution for VS-VTA pairs in seconds. In green the synchronous pairs. On the left, lag distribution for pairs detected in more precise time scale. Slight distribution asymmetry indicates directionality in the direction of $lag > 0$, meaning a predominance of pairs in which VS leads VTA. On the right, the lag distribution for pairs detected in the broader time scale, it presented as an asymmetric fat-tailed distribution.}
\label{fig:LagInSecAll}
\end{figure}
\begin{figure}[H]
\centering
\includegraphics[scale=0.48]{figures/LagSec4Typo3VS.png}
\caption{Lag distribution of four more represented pair-types in precise time scale. Only SPN-DAN pairs show a preferred directionality, namely $VS\rightarrow VTA$.}
\label{fig:LagInSec4typo}
\end{figure}
\begin{figure}[H]
\centering
\includegraphics[scale=0.48]{figures/LagSecLaser3VS.png}
%\includegraphics[scale=0.4]{figures/OnlyLaserOriz.png}
\caption{Lag distribution of laser tagged dopamine units in pair with SPN and FSN units, in precise time scale. Lag distributions are similar to the lag distributions of classified DAN in pair with SPN and FSN, confirming the directionality expressed by SPN-DAN pairs and the validity of the classification adopted in VTA.}
\label{fig:LagInSecLaser}
\end{figure}
The kind of activation exhibited by a directional pair is exemplified for an interregional directional pair in figure \ref{fig:directional_assembly}. The pair in example was detected with a bin width of $\Delta = 0.12$, and the inter-units activation lag was positive $lag = 0.36 sec$. On top is shown the mean across trials of the pair activity for rewarded (grey line) and unrewarded odor (purple line) in the original phase. From the activation profile and the separation between the two activation lines one can see that the illustrated assembly-pair specifically codes for the rewarded odor. On bottom are shown raster plot and mean firing rate of units in assembly. x-axis origin corresponds to the odor onset, while the red line marks the end of the stimulus duration.\\
Finally, looking at the activity of two neurons, the directionality exemplified by the pair is evident: first the VS unit becomes active and, after a temporal delay, the activation of VTA unit follows. Raster plots show how the activity across trials changes. In both units we see a gradient of activation across trials. In first trials (less than 10 trials) the activity is either very rare (VS unit), or uniform distributed in the odor window (VTA unit); after few trials, VS unit first, followed by VTA unit, show a peak in activation at the odor onset, that remains stable till the end of the phase.\\ 
\begin{figure}
    \centering
    \includegraphics[scale=0.6]{figures/DirectionalAsEx1.pdf}
    \caption{Directional assembly. On top mean across trails and standard error of the activity of the pair for rewarded (grey) and unrewarded odour (purple) in the original phase. On bottom raster plot and firing rate (mean and standard deviation) of units in assembly. x-axis origin corresponds to the odor onset, while the red line marks the end of the stimulus duration. The examined assembly has a positive lag, that means VS preceding in activity VTA, from neuronal activity we can see in fact the VS unit activate before than the VTA unit.}
    \label{fig:directional_assembly}
\end{figure}
At population level time scale and directionality segregation reveals different assembly-activation patterns. In figure \ref{fig:AsActBinLag} an example of dissection of activation patterns is shown through the heat map of interregional assembly-pairs activity averaged across trials in one of the paradigms of the experiment.\\Assembly-pairs exhibiting directionality $VS \rightarrow VTA$ and detected in the precise time scale ($\Delta \le 0.25$) became activated early during the odor presentation.\\The early activation at the stimulus presentation is the type of activation we expect when the animal predicts the reward. As discussed in \hyperref[sec:StateArt]{Section~\ref*{sec:StateArt}} (see figure \ref{fig:RewPred}) reward-related response in dopamine neurons change as the reward probability change, the typical prediction error signal comes as an activation at the stimulus presentation in case of high reward probability, and less activation at the retrieval. Moreover we have shown in figure \ref{fig:LagInSec4typo} that the SPN-DAN assembly-pairs are directional, thus we speculate that SPN-DAN are good candidate for reward prediction error coding; the investigation of this hypothesis will be argument of the next chapters.\\
\begin{figure}[h!]
    \centering
    \includegraphics[scale=0.56]{figures/AsActPerBinLag1.png}
    \caption{Assembly-activation patterns given time bins and lags. In a.) heat map of interregional pairs in one the experimental paradigms, b,c,d. pairs of a. selected for bin size ($\Delta$) and lag: $\Delta < 0.25 s$ and $lag > 0$ (b.), $\Delta > 0.25 s$ and $lag > 0$ (c.), $lag < 0$ (d.)}
    \label{fig:AsActBinLag}
\end{figure}
\begin{figure}[h!]
    \centering
    \includegraphics{figures/RewPred1.png}
    \caption{Predictive reward signal in dopamine neurons. Adapted from \cite{Fiorillo}}
    \label{fig:RewPred}
\end{figure}
\pagebreak
\section{Conclusion}
From the interregional time scales distribution we deduced that VS-VTA assembly-pairs consist in two classes of interactions with different time scales, highlighting so the complexity of VS-VTA pair interactions and underlines the importance to have chosen a method capable to explore different time scales.\\The two time scales involved in VS-VTA interaction were then separated to continue the study with the lag analysis, from which emerged a preferred directionality in the verse $VS\rightarrow VTA$, and specifically in pairs containing striatal projection neurons and dopamine neurons.\\Different time scales of interaction and directionalities reflect different patterns of assembly-pair activity. In a preliminary investigation we have shown that pairs detected at the precise time scale and exhibiting the directionality $VS\rightarrow VTA$ show a prominent activation during the stimulus presentation, as expected in case of prediction error signal.\\Since directional assemblies are formed by striatal projection neurons and dopamine neurons, we hypothesize that those pairs are good candidate for prediction coding. The next chapter is aimed to understand whether SPN-DAN assembly-pairs specifically compute prediction error. Comparing the activity of different assembly-pair types we ask whether different assembly-pair types specialize in different coding features.\\We start the analysis from the pair-types task related response. A systematic analysis of this is performed in \hyperref[sec:TaskResp]{Section~\ref*{sec:TaskResp}}.

 \section{Pair-types task-related pattern}
 \label{sec:TaskResp}
In the previous section we concluded that time scale segregation and directionality might reflect specific task-related coding feature. In a data subset we have shown how different time scales and directionalities reveal different assembly-pairs activity patterns.\\We have seen as well that different assembly-pairs types have different directionality distribution. SPN-DAN assembly-pairs, in particular, exhibit a clear $VS\rightarrow VTA$ directionality, so we expect those assembly to have different patterns of activity with respect to other assembly-pair types.\\Furthermore in this section we prove not only that different assembly-pairs-types have different task-related activity patterns, but also that they consequently express different coding functions.\\
To this purpose, we tested the responsiveness of each assembly-pair to the conditioned stimulus (CS $+/-$) and the unconditioned stimulus (US), through an analysis of variance performed using both a paired Friedman test and a non parametric version of repeated measure Anova test, on the trial by trial assembly-pair activity. The results of the two tests are consistent each other, we choose to present here only Friedman test results.
%leaving summary and comparison in\hyperref[app:FriedAnova]{~Appendix\ref*{app:FriedAnova}}.
The choice of non parametric tests was made to be free from gaussianity assumption.\\Time intervals were 0.5 $sec$ long, and were chosen as it follows: CS $+/-$ interval included all time steps $t \in [0, 0.5]$ $sec$ from the odor onset, CS$+/-$cont. was the window including each time $t$ with $t \in [-0.5, 0]$ $sec$ before the reward retrieval, in which the the odor was still present, US interval included times $t \in [0,0.5]$ from the reward delivery. The assembly-pair activity in the listed task relevant intervals was compared with the baseline interval including times $t \in [-0.8, -0.3]$ $sec$ before the odor onset, in which no task related activity could have interfered with the spontaneous neural activity.\\For each assembly-pair type we performed a large number of statistical tests, some will have p-values less than $0.05$ purely by chance, even if all null hypotheses of the family of tests performed are really true. Hence, to take this in account, Bonferroni correction for multiple comparison was applied. The Bonferroni correction is the classical approach to take in account the multiple comparison problem and it consists in having a control on the familywise error rate ($\alpha=0.05$). This is possible by setting the critical value $\alpha$ for an individual test at lower value than 0.05 in such a way that, if all the null hypotheses are true, the probability that the family of tests includes one or more false positives due to chance is 0.05. The value $\alpha$ for an individual test is obtained by dividing the familywise error rate by the number of tests.\\In figure \ref{fig:Baseline} is shown the heat map of the neural activity averaged across trials before and after the odor onset, during the baseline period (interval delimited by the two back lines) no activity patterns emerge besides the spontaneous neural activity.\\
\begin{figure}
    \centering
    \includegraphics[scale=0.6]{figures/Baseline.png}
    \caption{Baseline interval choice. Neural activity averaged across trials. For each trial the baseline interval was a time window in which no task related activity could have interfered with the spontaneous neural activity.}
    \label{fig:Baseline}
\end{figure}
Analysis of variance compares the means of several groups to test the hypothesis that they are all equal, against the general alternative that they are not all equal. This alternative was in our case too general. Hence, in the cases in which we obtained an omnibus significant $\chi^2$ test, the post-hoc procedure were designed using the matlab function $\textit{multcompare}$ whose critical values where computed with Bonferroni method (\cite{Bonferroni}, \cite{Dunn1958}, \cite{Dunn1961}).\\
\begin{figure}
    \centering
   % \includegraphics[scale=0.4]{figures/SPN_DANexStim1.png}
    \includegraphics[scale=0.38]{figures/SPN_DANexPreRew.png}
    
   \vspace{1cm}
   
   \includegraphics[scale=0.38]{figures/SPN_DANexStim1.png}
    \caption{Example of two assembly-pairs significant after the analysis of variance for which the post-hoc test shown that in both cases two groups out of four were significant different, respectively the baseline and the CS+ cont window (top) and the baseline and the CS+ window (bottom) }
    \label{fig:SPN_Ex}
\end{figure}
In figure \ref{fig:SPN_Ex} two examples of assembly-pairs resulting significant for Friedman test performed on hit trials (correct trials when the rewarded odor was presented).\\On top a SPN-DAN pair, for which the post-hoc analysis revealed a significant difference between the baseline and the CS+ cont. groups. Interestingly we observe, in CS+ cont. window, less assembly-activity in the first trials and intense assembly-activity in the last trials, the same trend is observed in CS+ window in another SPN-DAN assembly-pair example figure \ref{fig:SPN_Ex}.\\This dynamical responsiveness resembles a typical learning dynamic. In first trials the mouse lives an explorative period in which the rule of task has not been learnt yet, while it has been learnt at the end of the original phase. In other words, in the first part of the task the animal is almost always surprised when the reward is delivered, the expectation of reward is low, while during the second part of the original phase the mouse is able to predict the reward already at the stimulus presentation, the expectation of reward increases when the rewarded stimulus is presented.\\Moreover not debated considerations in literature about ventral striatum and dopamine neurons activity related to the expectation of reward (\cite{Schultz1992}, \cite{Schultz} \cite{Fiorillo}) show that in presence of probabilistic reward the response of dopamine units is peaked at the reward time when the probability to get the reward is low and is instead peaked at stimulus presentation when the probability to get the reward is high, VS neurons instead show a tonic response from the stimulus presentation and their neuronal activity is modified when subject learns to predict future rewards from sensory cues (\cite{Pagnoni}, \cite{Schultz2000}, \cite{Radua}).\\In this scenario, one can says that if an assembly-pair codes for reward prediction error, then the activity will be intense during the presentation of the rewarded stimulus only in the last part of the task-phase. An example of dynamic responsiveness across trials is shown in figure \ref{fig:SPN_Ex} (top) in the CS+cont. window, and in figure \ref{fig:SPN_Ex} (bottom) CS+ window. On the contrary in pairs significant at retrieval (US) we would expect intense activity in first trials, related to the surprise to have the reward, and less activation in the second part of task, when the animal acquired sureness about the reward. We will examine these learning transitions revealed in the variability across trials in\hyperref[sec:CorrRL]{~Chapter \ref*{sec:CorrRL}}.\\
The two examples in figure \ref{fig:SPN_Ex} above show that some assembly-pairs are implicated in the learning dynamic, with a systematic study in this direction we can understand how many of them are involved in the learning dynamic and which functions they express. Assuming that a good portion of assembly-pairs show a learning dynamic, following questions are: are they all involved specifically in prediction error computation? Do they compute differently the prediction error signal depending on their underlying units? To answer these questions and formulate hypothesis on the signals that can be formed in specific assembly-pairs, we need to recall how prediction error signals are expressed by dopamine neurons. Reward prediction error signal is formed by two components (\cite{Tobler2003}, \cite{Nomoto2010}, \cite{Fiorillo2013}, \cite{Schultz2016}): the first component consists in an early activation due to an unselective response to a large variety of unpredicted events included the potentially rewarded stimuli, that are however in this phase only detected, to be later identified and valuated.\\The detection component evolves then in the second component, also called main component, which defines the function of the dopamine response and reflects the evolving neuronal processing that is required to fully appreciate the value of the stimulus (see figure \ref{fig:probSchultz} in\hyperref[sec:StateArt]{~Chapter \ref*{sec:StateArt}}) The latter component is the biological implementations of reward prediction error term in reinforcement learning models.\\
Dopamine neurons express both these two components and different units do not exhibit different coding functions, this fact is important to ensure a robust and efficient coding \cite{Eshel}. Despite robust evidence for dopamine units involvement in prediction coding, it is hard to identify how these signals are encoded by the dopamine circuit. In this work we want to prove that although at single neuron level different dopamine neurons do not code specifically for one of the two components of prediction error (\cite{UchidaDop}) and they use the moment to moment reward prediction associated with environmental stimuli to compute a reward prediction error, the specificity of different components coding emerges in the assembly-pairs activity because different interactions convey different aspects of the prediction error coding. Thus dopamine neurons exhibit heterogeneous but robust response at population level, which would become specific in distinct VS interactions. The learning functions would be then, rather than cell-types-specific, dopamine-circuit specific (\cite{Saunders2018}).\\
We start to prove this hypothesis analysing the reward-related activity of assembly-pair types. We look for resemblance to reward prediction error signals, and we want to prove that the two prediction error components can be disentangled if we look at the specific VS interactions with dopamine neurons. In other words looking at the assembly-pair types rather than at the single dopamine response is a way to $"$orthogonalize$"$ the two prediction error components.\\
We focused the analysis on assemblies resulting significant for the Friedman test, the fraction in percentage of those assemblies is shown in figure \ref{fig:PercAsFried}. The histograms refer to the Friedman tests made on hit trials, in original and reversal phase separately.\\
\begin{figure}
    \centering
    \includegraphics[scale=0.5]{figures/PercFriedHitTrialsBFf.png}
    \caption{Number in percentage of assembly-pairs exhibiting a significant task related pattern activity. The Bonferroni correction for multiple comparison was applied. In A. the significant assembly-pairs tested for the original phase, in B. the significant assembly-pairs tested in reversal. Assembly are divided in the four principal pair-types.}
    \label{fig:PercAsFried}
\end{figure}
Assembly-pairs patterns of significant assemblies were plotted, and differences in assembly-types emerged. In figure \ref{fig:HeatPairsDan} we show assembly-pairs tested with Friedman test and resulting to have significant task related response in original phase. In order: A1. SPN-DAN assembly-pairs in original phase: a good portion of those pairs became active early at the rewarded stimulus (CS+) presentation, a good portion remains active during the stimulus presentation (CS+ cont.), until the reward retrieval (US), only a small fraction is active only in US window, this kind of answer makes those assembly-pair-types good candidates for prediction coding. In A.2 the same assembly-pairs in the reversal phase.\\FSN-DAN assembly-pairs on contrast have a phasic early response to the rewarded stimulus (CS+) and reward retrieval (US) (B.1); in B.2 same FSN-DAN pairs in the reversal phase, with respect to the original phase we notice less activation at the stimulus onset (CS+) and more at the retrieval.\\The early activation to the stimulus of FSN-DAN pairs leads to think that they are involved in the detection coding, however the response at the retrieval of FSN-SPN induces to believe that those assemblies are not involved in the reward prediction. One hypothesis could be that the detection component of prediction error is formed in FSN-DAN assembly-pairs, which evolves in the identification and valuation coding through the SPN-DAN circuit.\\Those consideration will be discussed in detail in \hyperref[sec:FalseAlCorrRej]{Section~\ref*{sec:FalseAlCorrRej}}, as well in \hyperref[sec:CorrRL]{Chapter~\ref*{sec:CorrRL}}, and \hyperref[sec:Regression]{Chapter~\ref*{sec:Regression}}.\\From figure \ref{fig:HeatPairsDan} and figure \ref{fig:HeatPairsGaba} we have seen that different assembly-types are active in different task moments, summary plots are shown in figure  \ref{fig:FriedHistoDAN} and figure \ref{fig:FriedHistoGABA} where we report in which windows (CS+, CS+ cont., US) assembly-pairs are significantly active with respect to the baseline. The total amount can exceed the $100\%$. For example, if one pair was found significant in the US window and CS+ cont. with respect to the baseline, this assembly-pair was double counted.\\All assembly-pair types are predominantly active in the window CS+ cont., that is the window of preparation for the reward.\\There are  differences among assembly-pairs types and phases. The first remarkable difference is between the original and reversal phase: in reversal phase for all pair-types we notice less activation in CS+ window with respect to the original phase. Assembly-pairs including SPN are in the reversal phase more active at retrieval (US). Assembly-pairs composed by GABA units became activated later than assembly-pairs composed by DAN, and they are broadly active before the reward and during the reward consumption.\\Differences between pairs with GABA and DAN were shown to highlight that assembly composed by putative neurons supposed to encode different signals, show indeed very different task related patterns.\\However, since we specifically want to investigate how prediction signals are formed in VS-VTA interaction, we will focus from now on only on the interregional assembly-pairs with dopamine neurons.\\
 \begin{figure}
     \centering
     \includegraphics[scale=0.32]{figures/HeatSPN_DAN.pdf}
     
     \vspace{1cm}
     
     \includegraphics[scale=0.32]{figures/HeatFSN_DAN.pdf}
     \caption{Assembly-pairs tested with Friedman test and resulting to have significant task related response in original phase. In order: A1. SPN-DAN assembly-pairs in original phase: a good portion of those pairs became active early at the rewarded stimulus (CS+) presentation, a good portion remains active during the stimulus presentation (CS+ cont.), until the reward retrieval (US), only a small fraction is active only in US window, this kind of answer makes those assembly-pair-types good candidate for prediction coding. In A.2 the same assembly-pairs in the reversal phase. FSN-DAN assembly-pairs on contrast have a phasic early response to the rewarded stimulus (CS+) and reward retrieval (US) (B.1); in B.2 same FSN-DAN pairs in the reversal phase, with respect to the original phase we notice less activation at the stimulus onset (CS+) and more at the retrieval.}
     \label{fig:HeatPairsDan}
 \end{figure}
 \begin{figure}
     \centering
     \includegraphics[scale=0.32]{figures/HeatSPN_GABA.pdf}
     
     \vspace{1cm}
     
     \includegraphics[scale=0.32]{figures/HeatFSN_GABA.pdf}
     \caption{Assembly-pairs tested with Friedman's test and resulting to have significant task related response in original phase. In order: in A.1 SPN-GABA assembly-pairs in original phase: a good portion of those pairs became active during the rewarded stimulus presentation (second half of CS+ and CS+ cont.), a good portion shows tonic activity until the reward retrieval (US). In A.2 the same assembly-pairs in the reversal phase. In B.1 FSN-GABA assembly-pairs are tonically active from the second half of the rewarded stimulus presentation window (CS+) to the reward retrieval window.}
     \label{fig:HeatPairsGaba}
 \end{figure}
 \begin{figure}
    \centering
    \includegraphics[scale=0.36]{figures/SPN_DANHisto.png}
    
    \vspace{1cm}
    
    \includegraphics[scale=0.36]{figures/FSN_DANHisto.png}
\caption{Percentage of assembly-pairs significant with respect to the baseline in the windows CS+, CS+ cont., US. The total amount can exceed the $100\%$ because if  one pair were found for example significant in the US window and CS+ cont. with respect to the baseline, this assembly-pair was double counted.}
    \label{fig:FriedHistoDAN}
\end{figure}
\begin{figure}
    \centering
    \includegraphics[scale=0.36]{figures/SPN_GABAHisto.png}
    
    \vspace{1cm}
    
    \includegraphics[scale=0.36]{figures/FSN_GABAHisto.png}
    \caption{Percentage of assembly-pairs significant with respect to the baseline in the windows CS+, CS+ cont., US. The total amount can exceed the $100\%$ because if one pair were found for example significant in the US window and CS+ cont. with respect to the baseline, this assembly-pair was double counted.}
    \label{fig:FriedHistoGABA}
\end{figure}\\
\pagebreak
\section{Hit, False Alarm and Correct Rejection Trials}
\label{sec:FalseAlCorrRej}
{\color{red} discussion is a draft}
In the previous session we studied the assembly-pairs task related response in Hit trials, namely trials in which the odor was rewarded and the mouse went for the reward. We pruned assembly-pairs using a Friedman test to keep only assembly-pairs with significant task-related activation, in specific moment of the trial.\\In similar way we analyse the assembly-pairs task-related response for False Alarm trials (unrewarded odor, the mouse went for reward) and Correct Rejection trials (unrewarded odor, the mouse sat quiet).\\The three windows of interest were re-set up for these trials types in the following ways: in False Alarm trials those windows are called CS-, CS- cont., 3rd Lick. The CS- window, analogously to the CS+ in Hit trials, is a time interval of 0.5 $sec$ starting from the stimulus onset (unrewarded in this case); the CS- cont. and 3rd Lick are respectively the time intervals, again 0.5 $sec$ long, immediately before and after the expected reward.\\We used the following assumption to establish the expected reward time: since when the rewarded odor is presented the mouse has to lick at least two or three times depending on the paradigm to get the reward, when he goes for the reward even though the odor is unrewarded, he expects the reward after the second or the third lick depending on the paradigm used.\\In Correct Rejection trials the animal does not expect the reward, however we used as $"$hypothetical reward$"$ time, the average reward time obtained from Hit trials, to compare the assembly-pairs activity before and after the $"$hypothetical reward$"$ time with the activity before and after the reward time (Hit trials) or the expected reward time (False Alarm trials).\\The comparison between Hit, False Alarm and Correct Rejection is interesting because we expect different VS-VTA assembly-pair patterns in relation with reward prediction and reward occurrence, indeed dopamine is thought to be a key regulator of learning from appetitive as well as against-reward events (\cite{Schultz1997}, \cite{Wenzel}, \cite{Fiorillo2013b}, \cite{Schultz2015}).\\When the animal receives reward unexpectedly, dopamine neurons fire a burst of action potentials. If a sensory stimulus reliably predicts reward, however, dopamine neurons decrease their response to reward, and instead burst to the stimulus, and finally if an expected reward is omitted, dopamine neurons pause their firing at the time they usually receive reward. In\hyperref[sec:StateArt]{~Chapter \ref*{sec:StateArt}} the scheme in figure \ref{fig:RewardDoya} resume the principal dopamine reward-related responses.\\
We assume that assembly-pair task related patterns reflect the predictive coding of underlying units, furthermore we expect that assembly analysis could reveal how predicted signal is formed in VS-VTA circuit, that is in fact still discussed and unclear.\\Recent works (\cite{UchidaDop}) shown that dopamine neurons exhibit homogeneity in response to both unexpected and expected rewards. Such uniformity ensures robust information coding, allowing each dopamine neuron to contribute fully to the prediction error signal. In recent years relative new hypothesis on the dopamine circuit specificity (\cite{Takahashi2016}, \cite{Saunders2018}) emerged, they remains however not fully proved. According to these assumptions, a specialization in different aspects of prediction error could be expressed through interregional patterns of coherence in the VS-VTA circuit. If the hypothesis of a specialized code is true, we would observe differences in activity patterns of VS-VTA assembly-pairs composed by dopamine units and different cell-types in VS.\\Interestingly we found that signals presented in the scheme \ref{fig:RewardDoya} are present both FSN-DAN pairs in hit trials (figure \ref{fig:HeatFSN_DANComp}, panels A.1, A.2), and SPN-DAN pairs (figure \ref{fig:HeatSPN_DANComp},panels A.1, A.2).\\Despite at first sight both assembly-pairs type seem to be involved in prediction coding, we want here emphasize the importance of the two components of the reward prediction-error signalling introduced in the previous session (\cite{Tobler2003}, \cite{Nomoto2010}, \cite{Fiorillo2013} \cite{Schultz2016}), and endorse the hypothesis that SPN-DAN pairs and FSN-DAN pairs are involved in different components of reward prediction coding.\\
In the previous session we introduced the two sequential components of neuronal response, called detection and identification/valuation, suggesting that sharp assembly-pairs FSN-DAN most probably encode a generic detection of the stimulus, whereas SPN-DAN assembly pairs, characterized by broader response could encode the valuation, that is the component associated to the prediction error in reinforcement learning models.\\
\begin{figure}
    \centering
    \includegraphics[scale=0.4]{figures/Schultz2016CSMod.png}
    \caption{Adapted from \cite{Schultz2016}. Voltammetric dopamine responses in rat nucleus accumbens distinguish between a reward-predicting conditioned stimulus (CS+) and a non-reward-predicting conditioned stimulus (CS-). The dopamine release comprises an initial indiscriminate detection component and a subsequent identification and value component.}
    \label{fig:dopCS}
\end{figure}
SPN-DAN assembly-pairs show similar activity-patterns in Hit and and Correct Rejection trials (figure \ref{fig:HeatSPN_DANComp}), however much less assembly-pairs show significant task-related response in Correct Rejection trials, SPN-DAN assembly-pairs preferentially code for the rewarded odor. Interestingly in False Alarm trials we observed less stimulus activation with respect to the two correct realizations (Hit and Correct Rejection trials). This fact is guessed: the animal is indeed unsure about the stimulus received and performs wrong, while in Hit trials, the animal is more confident\footnote{here a distinction between the early and the final stage of the task would be due, we leave this discussion to the next chapter.} about the output will be received, then, being able to predict the reward, reacts to the conditioned stimulus, this fact is again in agreement with the prediction error coding. In figure \ref{fig:dopCS} a useful comparison between neural responses in presence of rewarded and unrewarded stimulus. Again resemblance with the identification and valuation component suggest that SPN-DAN assembly-pairs encode the main component of the prediction error signal. As shown in figure \ref{fig:dopCS} the mean response is anticipated in case of CS-, and again when CS- is presented often right after the identification a depression occurs. The amplitude of response cannot be judge by the task related response, because for visualization purpose the activity was normalized over the maximum of each of three trials realization (Hit, Correct Rejection and False Alarm trials)\\FSN-DAN assembly-pair activity patterns are more diversified among trials realizations, coherently with the hypothesis that they are unselective to the stimulus type and multisensory in response. In Hit trials (figure \ref{fig:HeatFSN_DANComp}, boxes A.1, A.2) the stimulus (CS+) leads an early activation, this activation can be associated with the initial component of the reward prediction error, namely the detection. In hit trials more than $\% 50$ of FSN-DAN pairs phasic activate at retrieval (US), suggesting that those pairs are not able to predict the reward. In False Alarm trials (figure \ref{fig:HeatFSN_DANComp}, boxes B.1, B.2) the the stimulus presentation stimulus, unrewarded in this case (CS-), leads an activation of the majority of FSN-DAN pairs right after the stimulus presentation. In the windows before and after the expected reward (CS- cont., 3rd Lick) a good portion ($\sim50\%$) of FSN-DAN assembly-pairs show depression when the reward was predicted but it does not occur, probably coding for the frustration for realizing there will not be a reward, or some negative emotion.\\FSN-DAN assembly-pairs reward-related answer is rather unspecific and difficult to interpret. However FSN-DAN pairs could be specific in other aspects of mediation of experience-dependent behavior, such as the motivational salience either at the odor presentation, or at the choice. The assembly-pairs significant in original phase keep the same activation pattern in reversal with only slightly differences.\\One could observe that a similar inhibition was observed also in dopamine neurons when the reward was predicted and no reward occurred (see figure \ref{fig:RewardDoya1}, panel A, bottom), however in classical dopamine neurons activity the depression occurs in correspondence of the expected reward, while in FSN-DAN assembly-pairs the inhibition is anticipated in False Alarm trials, suggesting that they are involved in a different type of coding that is not (or only partially) represented by the classical reward prediction error definition used in reinforcement learning models.\\Finally we show FSN-DAN activity patterns in Correct Rejection trials (figure \ref{fig:HeatFSN_DANComp}, boxes C.1, C.2): here we observe that a good portion of FSN-DAN assembly-pairs is early positively activated by the stimulus presentation (CS- window), another group ($\sim30\%$ of assembly-pairs) is instead inhibited. This depression can be interpreted as codification of the disappointment of the animal when the recognized unrewarded odor is presented, furthermore the $\sim10\%$ of assembly-pairs show an inhibitory response in CS- cont. window, after an excitation in CS- interval.
\begin{figure}
\centering
\includegraphics[scale=0.32]{figures/HeatSPN_DANHit.png}
\includegraphics[scale=0.32]{figures/HeatFA_SPN_DAN1.png}
\includegraphics[scale=0.32]{figures/HeatCR_SPN_DAN1.png}
\caption{SPN-DAN assembly-pairs significant for Friedman test in three trial realizations (Hit, False Alarm and Correct Rejection). SPN-DAN pairs show similar pattern of activity in the three realizations, however it is worth to remark in False Alarm trials less activation at the stimulus presentation and it appears a group of pairs at the expected reward time. In Correct Rejection trials much less assembly-pairs are significant. SPN-DAN assembly-pairs preferentially code for the rewarded odor. Of those significant assembly in Correct Rejection trials$\sim 80\%$ of pairs are early activated by the stimulus (CS-).}
\label{fig:HeatSPN_DANComp}
\end{figure}
 \begin{largefigure}[16pt]
 \centering
 \includegraphics[scale=0.27]{figures/HeatFSN_DANHit.png} \includegraphics[scale=0.27]{figures/HeatFA_FSN_DAN1.png}
 \includegraphics[scale=0.27]{figures/HeatCR_FSN_DAN1.png}
  \caption{FSN-DAN assembly-pair activity patterns are diversified among trials realizations. In Hit trials (A.1, A.2) early stimulus (CS+) response and a phasic response on retrieval (US), in False Alarm trials (boxes B.1, B.2) the activation to the stimulus remains, no excitatory activation in the windows before and after the expected reward time. Instead we observe a good portion ($\sim50\%$) of FSN-DAN assembly-pairs showing inhibitory activity with respect to the baseline. In Correct Rejection trials (boxes C.1, C.2) a good portion of FSN-DAN assembly-pairs is excitatory activated early by the stimulus presentation (CS- window), $\sim30\%$ of assembly-pairs exhibits inhibitory response, the $\sim10\%$ of assembly-pairs shows an inhibitory response in CS- cont. window, after an excitation in CS- interval.}
  \label{fig:HeatFSN_DANComp}
\end{largefigure}
\begin{figure}
    \centering
    \includegraphics[scale=0.5]{figures/RewardDoyaRLSUM.png}
    \caption{Adapted from \cite{Doya}. In the first two panels response associated to rewarded trials. Last panel is related to unrewarded trials, in the cases in which the reward was predicted but it did not occur. Activity of the dopamine neuron is depressed exactly at the time when the reward would have occurred.}
    \label{fig:RewardDoya1}
\end{figure}
\pagebreak

\section{Conclusion}
{\color{red} this part is a draft. Rephrase properly hypothesis!!!! clear hypothesis on SPN-DAN and FSN-DAN need to emphasized }
In the previous two sections we extensively described the task-related activity of assembly-pair types. We first noticed that different assembly-pairs types show different task-related patterns of activity, and consequently they could exhibit different coding feature.\\The observed activity patterns, at the stimulus presentation and at the retrieval, let to suppose that VS-VTA pairs are able to integrate the difference between expected and received outcomes.\\ Typical single neuron activity in VS and VTA related to expectancy of reward and uncertainty was largely investigated in last decades (\cite{Fiorillo}, \cite{Schultz}, \cite{Schultz1992}, \cite{Schultz1998}), however how the VS-VTA interaction could form and encode those kind of signals is not fully understood. As first step we made a parallelism between the single unit activity related to probability of reward and the activity exhibited by the detected assembly-pairs.\\From the resemblance of their response we have been able to assume that SPN-DAN assembly-pairs encode prediction signals.\\Despite FSN-DAN pairs seem to not code for the second component of prediction error, they show 
interesting behavior in False Alarm trials, where $\sim 50\%$ of FSN-DAN pairs show inhibitory responses before the rewarded was expected, from which one can guess the $"$disillusionment$"$ coding, whereas in Correct Rejection trials $\sim 30\%$ of FSN-DAN pairs shows an inhibitory response already at the presentation of the recognized rewarded odor, from which one can guess an ability of those pairs to code for the $"$disappointment$"$ related to the consciousness of not getting the reward.\\FSN-DAN pairs seem to be involved in prediction encoding in a different way that SPN-DAN pairs, moreover they unspecifically react to the stimulus in response. The unselective and multisensory nature of their response could correspond to a very braod range of coding futures that is partially explained by detection component of the reward prediction error definition when they activate at the stimulus onset. Their phasic activation at retrieval, as well the late inhibition in False Alarme and Correct Rejections trials suggest an involvement in some outcome-related feelings or choice (lick, no-lick) execution-related activation, which reflects in turn motivational components, in agreement with what is suppose to be the role of ventral pallidal (VP) units, being VP considered in a unique position to process motivationally-relevant stimuli and coherent adaptive behaviors \cite{Root}. Motivation components are barely explained by the classical reward prediction error signal as parameterized in reinforcement learning model.\\From SPN-DAN pair activity patterns and FSN-DAN activity patterns, it emerges a clear specificity of assembly-pairs, suggesting that different assembly-pairs specialize in different aspects of prediction error, whereas was tested that at single units level responses of dopamine neurons to both expected and unexpected rewards follows the same function (\cite{UchidaDop}).\\Next chapter is entirely dedicated to prove the hypothesis of specificity of interregional pairs in reward related signal. In particular we asked whether SPN-DAN or FSN-DAN pairs are involved in prediction coding during a learning process, and if they share common response function for reward prediction error.\\The analysis shown until this point is static and cannot fully answer this question. The prediction signal is considered to be the basis of associative learning (\cite{RescorlaWagner}), and bears a striking resemblance to machine learning algorithms (\cite{SuttonBarto}). Thus, to address the aforementioned question we pass to a dynamic vision introducing the learning dynamic through a machine learning model.

 %%%%%%%% Commento forse da buttare
 %First difference between original and reversal phase is number of assemblies responding during the post-stimulus interval, this number decrease for all the typologies involved during hit trials (as shown in figure  (\ref{fig:histo_taskrel})), in according to the intuition, in reversal phase in fact, when the animals became more expert, neuronal response tend to be shifted closer to the reward delivery. This effect can be seen in figure  (\ref{fig:NeusInAsse}) where activity and raster plots of two units in a pair are shown. Looking at the raster plots, a shift in correspondence to the phase-switch is evident in both units.
 %\begin{figure}
  %   \centering
  %   \includegraphics[scale=0.3]{figures/Original_Hit_N.jpg}
    % \includegraphics[scale=0.3]{figures/Reversal_Hit_N.jpg}
     %\caption{{\color{red}TO MODIFY!!!! BUT THIS IS THE PLOT}}
     %\label{fig:histo_taskrel}
 %\end{figure}
%\begin{figure}
  %  \centering
   % \includegraphics[scale=0.6]{figures/SingleNeus1_15Lastrev1Pru_An_4.jpg}
    %\caption{Shift in time of neuronal activity of two units in assembly. }
    %\label{fig:NeusInAsse}
%\end{figure}

%\section{Discussion}
%\section{Combination of single neuron and assemblies analysis}
%\subsection{Directionality using classification}
%\subsection{Significant task related response for typology}
%%%%%%% Fine Commento forse da buttare


%%%%% Commento Utile
%To better study assemblies activation patterns, first the task relevant moments of the experiment were selected. From the mean task related activity patterns we expected to see differences among assemblies types in two experimental chapters (original and reversal). To better visualize the task related activation patterns via heat plots, hit trials (rewarded odor, mouse went for reward), correct rejection trials (unrewarded odor, mouse sat quiet), false alarm trials (unrewarded odor, mouse went for reward), were kept separated; however this separation among trials types was released in further analysis, without affecting results.
%The assemblies were pruned according their significant task related activity, that was tested with Friedman's test and a non parametric version of the repeated measures Anova. We preferred to use non-parametric tests to be free from the assumption of gaussianity of the observations. Results of the two tests were consistent each other. The two relevant events of the task were the odor onset and the reward delivery, then we choose whether the assemblies showed a significant activity in three windows: Stimulus [0s, 0.5s], Pre-Reward [-0.5s, 0s], Reward [0s, 05s], the baseline was chosen in the interval [-1s, -0.5s] from the odor onset. Post-hoc analysis were performed using the Bonferroni's criterion {\color{red}check whether the criterion was Bonferroni of some other}. Almost $80\%$ of the VS-VTA assemblies showed a task related activity significant different from the baseline or from another of the windows considered. Of the significant assemblies $\%$ were composed by MSN-Type I units, $\%$ by FSI low-Type I, $\%$ by FSI high-Type I, $\%$ MSN-Type II, $\%$ by FSI low-Type II, $\%$ by FSI high-Type II, the other possible units combinations constitutes a minority and all toghether were the $\%$ {\color{red} Insert numbers of percentage}.

%\section{Conclusion}
