\chapter{Overview}
\label{chap:Overview}
\section{Motivation}
Neuronal models of reinforcement learning assume interactions of midbrain dopaminergic neurons and striatum to compute the differences between anticipated and received outcomes. The nature of this cross-areal interaction is however not fully understood. On this purpose we recorded dual site simultaneous electrophysiological in-vivo data from Ventral Striatum, including Pallidum, (VS) and Ventral Tegmental Area (VTA). On this data set we applied a cell assembly detection algorithm (\cite{RussoDurstewitz}).
The novel statistical approach presented treats the temporal scale and precision of coherent activity patterns as free parameters, to be determined from the data, thus, instead to make assumption, we deduced from data temporal scales and precision involved in assemblies pairs with units either from both regions or from only one of the two regions, shedding light on VS-VTA interaction temporal structure and VS-VTA directionality.
We can use the term directionality in our case because we use the algorithm at level of inter-regional pairs. It is important to recall that the cell-assembly algorithm returns lags between the units activation in assembly, thus using the algorithm at level of inter-regional pairs provides us the lag between two units activation, of those two units one is in VS and another in VTA, a lag in activation of such kind of pair, indicates which region is preceding the other in activation.
Our nomenclature was such that a positive lag means that VS is prior in activation, a negative lag, vice-versa, means that VTA is preceding the activation of VS.
Taking advantage of well defined neuron typologies classifications both in VS and VTA, we further investigated the specific cell-type composition of the assemblies exhibiting directionality. 
{\color{red}da finire}
%Using the bin size distribution
\section{State of the art}
 In recent decades Ventral Striatum and Ventral Tegmental area have been widely studied by scientists because of their prediction coding feature.
 To introduce the study on VS-VTA interaction an overview on the state of the art of the Ventral Striatum and Ventral Tegmental area knowledge is needed.