\chapter{Overview}
\label{chap:Overview}
\section{Motivation}
\label{sec:Motivation}
The mesolimbic dopamine pathway comprising the ventral tegmental area and projection terminals in the striatum has been identified as a critical neural system involved in processing both the rewarding and aversive behavioral effects of rewarded and unrewarded stimuli.\\Neuronal models of reinforcement learning assume interactions of midbrain dopaminergic neurons and striatum to compute the differences between anticipated and received outcomes (\cite{Schultz2001}, \cite{Schultz2002}, \cite{Fiorillo}, \cite{Eshel1}, \cite{Pagnoni}, \cite{Radua}, \cite{Takahashi2016}). The nature of this cross-areal interaction is however not fully understood.\\In this work we propose an in depth study on the formation of prediction error signals in interregional assemblies during reinforcement learning. On this purpose we recorded dual site simultaneous electrophysiological in-vivo data from ventral striatum, including Pallidum, (VS) and ventral tegmental area (VTA). On this data set we applied a cell assembly detection algorithm (\cite{RussoDurstewitz}).
The novel statistical approach is free to detect spike patterns at any time scale and coordination enabling so the investigation of the time scales and the inter-units lag activation involved in those interregional interactions.\\One of the most puzzling question about VS-VTA interactions is if there exists a preferred direction in which the prediction error signal is encoded. Restricting the analysis of assemblies at level of assembly-pairs we investigate the directionality between VS-VTA interactions through the inter-unit lag activation. Using the algorithm at level of inter-regional pairs the lag in activation between two units, one in VS and the other in VTA, is nothing but the lag in activation between the regions. Our nomenclature is such that a positive lag means that VS is prior in activation, a negative lag, vice-versa, means that VTA is preceding the activation of VS. VS predominantly leads VTA in detected assembly-pairs.\\Moreover interregional assembly-pairs show a bimodal time scale distribution, such bimodality is solely present in VS-VTA pairs and does not emerge in intraregional pairs. Looking at the assembly-pairs activity it emerges that different time scales and directionalities dissect different activity patterns, in particular directional assembly-pairs with VTA following VS in shorter time scale show activity patterns in agreement with prediction error encoding (\cite{Tobler2003}, \cite{Nomoto2010}, \cite{Schultz2016}).\\Taking advantage of well defined neuron typologies classifications both in VS and VTA, we further investigate the specific cell-type composition of the assemblies exhibiting directionality. Interestingly only assembly-pairs formed by putative striatal projection neurons (pSPN) and dopamine neurons (pDAN) are directional in the direction of VS leading VTA.\\Thus, considering the dissection in different activity patterns brought by different time scales and directionality, we assume that different assembly-pair types are diversified in task related activity. We examine the task related patterns of different assembly-pair types in three windows of interest, defined from crucial moments in the task proposed, the stimulus and the reward. Significant assembly-pairs activity will be assessed by Friedman test.\\The difference in assembly-pairs types task related patterns reflect different coding and encoding features. Based on broad evidence (\cite{Eshel}) dopamine neurons share common response function for reward prediction to guarantee robust information coding inasmuch each dopamine neuron contributes fully to the reward prediction error. However is not proved whether different neuronal interactions involving dopamine neurons specialize in different aspect of reward prediction error. In the last years it has been shown indeed that reward prediction is consisting of different components (\cite{Nomoto2010}, \cite{Fiorillo2013b}, \cite{Schultz2016}). The first, detection component, reflects a unspecific response to stimulus, in order to detect it regardless its association with the reward. After this phase the stimulus is identified in order to assign to it a value, allowing the animal to predict the reward. This second component of prediction error, also called the main component, constitutes biological implementations of the crucial error term for reinforcement learning according to the Reinforcement Learning models. The hypothesis we will make is that the main component signal is formed specifically by SPN-DAN pairs.\\We build this hypothesis starting from assembly-pair types task related patterns, precisely. SPN-DAN assembly-pairs response resemble to reward prediction error signals, that are not equally found in other assembly-pair types. From the evolution in trials of the assembly-pairs response, it is clear that the assembly get modified by the high dynamic of task, proper of the learning process.

\section{State of the art}
\label{sec:StateArt}
 In recent decades ventral striatum and ventral tegmental area have been widely studied by scientists because of their prediction coding feature.
 To introduce the study on VS-VTA interaction an overview on the state of the art of the ventral striatum and ventral tegmental area knowledge is needed.
\section{Experiment and data set}
\label{sec:Dataset}
In this session we describe the experiment and the collected data for the analysis in object in this work. The experiment and the single units classification was conducted by Max Scheller, medical student, member of the Kelsch group.\\
The task presented is a go/no-go reversal odor discrimination task (see details in \hyperref[sec:MatAndMet]{~Section\ref*{sec:MatAndMet}}). In figure\ref{fig:experiment} the scheme of the experimental set-up. Dual-site in-vivo data were recorded in ventral striatum and ventral tegmental area, while two odors were presented to head-fixed mice, one rewarded (CS+) with reward probability 0.9, the second one unrewarded (CS-). Learning the task consisted in licking at least a defined number of times (two or three depending on the paradigm) during a specific interval, called lick window, when the rewarded odor (CS+) was presented (hit trials) and not licking when unrewarded odor (CS-) was presented (correct rejection trials). Whereas a bad performance consisted in not licking, or licking outside the lick window period, at CS+ presentation (miss trials), or licking at CS- presentation (false alarm trials). Once the performance criterion was reached, the contingencies were switched, in other words the rewarded odor became unrewarded and vice-versa. Once the performance criterion was reached in the reversal phase the extinction phase followed, in which neither one of the two odors was rewarded. The lick window was open for an interval ranging from 1500 $ms$ to 2000 $ms$ depending of different paradigms. after a delay from 500 $ms$ to 1500 $ms$. Only licks happening in the lick window interval were counted as valid to get the reward.\\
\begin{figure}
    \centering
    \includegraphics[scale=1]{figures/Experiment.png}
    \caption{Scheme of the experimental setup. Two odors in randomized sequences were presented, the mouse was head-fixed and, to get the reward, had to lick during the allowed licking period when the rewarded odor was presented. Electrodes were implanted in ventral striatum and ventral tegmental area to record the neural activity. }
    \label{fig:experiment}
\end{figure}

\begin{figure}
    \centering
\includegraphics[scale=1]{figures/Performance.png}
\caption{Example of one animal's performance in original and reversal phase. In the shown paradigm the performance criterion to be reached to switch to the reversal phase was $79\%$, meaning that this level of accuracy had to be satisfied for hit trials and correct rejection trials. Black line indicates global performance including hit trials and correct rejection trials. Green line is the performance for the performance in hit trials and red lines is the performance in correct rejection trials.}
\label{fig:performance}
\end{figure}
\section{Materials and Methods}
\label{sec:MatAndMet}
\subsection{Data acquisition}Neural data was recorded with a 64-channel headstage (RHD2164) and interface board (RHD2000) by Intan Technologies LLC at 30 kHz sampling frequency. The same interface was used to record stimulus onset and offset, reward application, licking activity and laser stimulation.
\subsection{Probe Design}
Given the relatively high distance ($>6 mm$) between sites, depth of the target structures ($>5 mm$) and curvature of the OT/VP,  we developed a probe design specifically attuned to these challenges and easily adaptable to different single- or multi-site configurations. A detailed tutorial for the streamlined successor version of the implant used here is in preparation. The design exploits the high precision of the mature production technology of printed circuit boards (PCB)  ($\pm10\mu m$, custom-designed boards ordered from Wuerth Electronics) to achieve a precise x-y-placement of the guiding channels for optical fibers (Thorlabs \#) and tetrodes (spun from $12.5 \mu m$ teflon-coated tungsten wire, California Fine Wire).  For the VTA, four tetrodes were attached to an optical fiber and fixed in their guiding channel.  For the OT/VP, the implant was placed above a mould of the ventral brain surface, and tetrodes and an optical fiber were advanced to their z-position and fixed in place, matching the bowl-shaped 3d-curvature of the OT (see pictures). Single tetrode wires were attached to the PCB using gold pins (Neuralynx). The connection between PCB and head stage was made via a Molex SlimStack connector (mated height: 1 mm, pitch: 0.4 mm, 70 channels, \# $502426-7030$) and an custom adaptor to the 2x36 Omnetics Nano Strip connectors of the Intan RHD2164 head stage.\\ 
\begin{figure}
    \centering
    \includegraphics[scale=0.4]{figures/Implant.png}
    \caption{Foto provided by Max Scheller.}
    \label{fig:implant}
\end{figure}
\subsection{Animals and surgery details}
18 adults mice(DAT:Cre$+$, ChR2:YFP$+$ )(5-6 months aged) were implanted. The surgery duration was less than 2 hours, after which mice were housed individually. The recovery period was one week.\\
After termination of the experiments, animals were deeply anaesthetized with isofluorane and transcardially perfused with [x$\%$] formaldehyde solution. Placement of tetrodes in the target areas was evaluated post mortem by first explanting the whole brain, dorsal skull and attached probe as a unit, documenting the ventral aspect, where tetrode tips could be seen through the surface of the OT (foto\ref{fig:surgery}, left), and, in some cases, breaching it (foto\ref{fig:surgery}, right). After that, the relevant hemisphere was sectioned ($100 \mu m$) sagitally with a microtome (model \#) and mounted serially, so tetrode tracks could be traced.
\begin{figure}[h!]
    \centering
    \includegraphics[scale=0.2]{figures/surgery_3.jpg}
    %\includegraphics[scale=0.1]{figures/Surgery2.png}
    \caption{Foto provided by Max Scheller}
    \label{fig:surgery}
\end{figure}
\subsection{Behavioral conditioning in the go/no-go task.} We trained the animals in the head-fix setup described above. Mice received water in their home cage so that their body weight stabilized at 85$\%$ of baseline body weight. The training comprised multiple stages and progressed after reaching a performance criterion defined as at least 80$\%$ correct responses in 50 consecutive trials. Trials were considered correct if either at a $'$go$'$-response the reward was retrieved or at a $'$no-go$'$-response no significant licking was detected during the lick window. In the initial sessions, the animals’ licking behavior was shaped by first presenting them with a drop of water and subsequently letting them obtain more water when they licked at the licking spout (available in a random interval schedule, 0.5-12 s). Stage 1 ($'$training$'$): A single odorant (1.5 s stimulus duration) was presented. Animals could obtain a 5 $\mu l$ drop of water if they licked at least three times during a window from 0 to 2.5 s after stimulus onset (‘retrieval window’), this was considered a $'$go$'$-response. The interval between trials was randomly set at $10\pm 2$ s in all stages. Stage 2 ($'$discrimination$'$) consisted of two odorants in pseudo-random succession (no more than 3 consecutive trials with the same stimulus). One odorant (1.5s duration) was rewarded as in stage 1 (retrieval window: 0.5 – 2.5 s), while a $'$go$'$-response for the second odorant was registered as a false alarm and thus incorrect. No punishment was used. Stage 3 ($'$reversal learning$'$) used the same parameters as stage 2, but upon reaching the performance criterion (in the $'$original phase$'$), the reward contingency of the odors was switched ($'$reversal phase$'$). The data set used in this study consists of one reversal learning session per animal (10 sessions with a total yield of 75 single units) after we accustomed them gradually to a longer lick delay (1.5 – 3 s), keeping odorant delivery concurrent (3 s duration). 

\subsection{Data pre-processing: Spike detection.} To reduce noise and movement artifacts affecting all recording sites, we subtracted the median voltage trace of all channels from each recorded trace. The resulting signal was band pass filtered between 300 and 5000 Hz (4th order Butterworth filter, built-in MATLAB function). A threshold value for spikes was computed as a multiple (7.5x) of the median absolute deviation of the filtered signal (\cite{Quiroga}). Temporally proximal detected peaks over threshold were pruned by height to a minimum distance of 1 ms to avoid multiple detections of the same multiphasic spike. When an event was detected on multiple channels of a tetrode, the timestamp of the highest detected peak was used. Spike waveforms were extracted around $-10$ to $+21$ samples around the peak.

\subsection{Data pre-processing: Spike sorting.} Spike sorting was done with a custom-built graphical user interface in MATLAB, originally developed by A. Koulakov (CSHL). Metrics used for clustering included detected peak height or amplitude (and the respective principal components over channels), and the first three principal components of the waveforms for each respective channel when a spike was predominantly recorded on one channel. The quality of single unit clusters was assessed using the mlib toolbox by Maik Stuettgen (Vs. 6, $\href{https://de.mathworks.com/matlabcentral/fileexchange/37339-mlib-toolbox-for-analyzing-spike-data}{MLIB-toolbox-for-analyzing-spike-data}$) with particular attention to peak height distribution (fraction of lost spikes due to detection threshold), contamination (fraction of spikes during the refractory period $<5ms$) and waveform variance.\\
%For analysis of the silicon probe data, spike detection and spike sorting was done using KiloSort (\cite{Pachitariu}), followed by manual curation with Phy (\href{https://github.com/kwikteam/phy}{KiloSort}) using similar parameters to assess unit quality.\\
\subsection{Data pre-processing: Inclusion criteria of units.} For further analyses, units were only included if they complied with a set of criteria: Throughout the analyzed part of the recording session units were allowed to only have a maximum change in baseline firing rate from beginning to the end of the session of less than ten percent and intermittent maximum fluctuations of 20$\%$.\\ After exclusion of the first ten trials of odorant application where we frequently observed an initial response habituation, both odor responses had to be stable throughout the $'$pre$'$ phase.
\subsection{Optogenic tagging}
Identification of optogenetically modulated units was achieved by crosscorrelation of spike train and laser pulses.\\
After sessions, trains of 5 ms laser pulses (8-12 Hz, 5 mW) were delivered in either brain region via the implanted optical fibers. For each unit/region-pair, a test crosscorrelogram was computed from the timestamps of spikes and laser pulses (bin width: 1 ms, lag:  0-20 ms). To test for significant modulation, control crosscorrelograms ($n = 10000$) were constructed by shifting each laserpulse randomly in the interval $\pm$30ms around its original time, thus destroying any temporal relation on that timescale while preserving the properties of the spike train. Modulation was considered significant if two consecutive bins of the test crosscorrelogram lay outside of the middle 95$\%$ of the global distribution in the control crosscorrelograms. 
