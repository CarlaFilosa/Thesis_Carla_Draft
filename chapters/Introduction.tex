\chapter{Introduction}
\label{chap:Introduction}
%\section{Motivation}   
%\label{sec:Motivation}
The mesolimbic dopamine pathway comprising the ventral tegmental area (VTA) and projection terminals in the ventral striatum (VS) has been identified as a critical neural system involved in processing both the rewarding and aversive behavioral effects of rewarded and unrewarded stimuli. In these functions the brain performs simple arithmetic: it compares the expected and the received outcomes and computes the differences between the two. In other words after receiving the outcome, it computes the error made in predicting the outcome. This signal is called reward prediction  error (RPE): essential to learning, to reward maximization and guiding so reward-related behavior, RPE has been widely investigated in last decades. In dopamine neurons RPE signals are characterized by activations following primary food and liquid rewards, and visual, auditory and somatosensory reward-predicting stimuli. The reward-related activation is usually preceded by a brief detection component before the stimulus has been identified and properly valued, which evolves in time in the valuation component where the stimulus is fully appreciated (\cite{Schultz2001}, \cite{Schultz2002}, \cite{Fiorillo}, \cite{Eshel1}, \cite{Pagnoni}, \cite{Radua}, \cite{Takahashi2016}, \cite{TianHuang}). However, how the RPE computations are implemented in VS-VTA neural circuits remains elusive.\\
%Previous modeling and experimental studies have shown that simple arithmetic computations may arise from a wealth of nonlinear mechanisms to transform synaptic inputs into output firing at the level of single neurons ( Chance et al., 2002; Holt and Koch, 1997; Silver, 2010 ). However, it remains unknown whether these mechanisms underlie brain computations in a natural, behavioral context.\\
%The VTA receives indeed inputs from many different brain regions and cell-types, but what each of these inputs contributes to RPE computation is unclear. 
%Recently Watabe-Uchida and colleagues have shown that these input neurons show a wide range of different response profiles to cues and rewards (\cite{TianHuang})
Thus, understanding VS-VTA related circuits, will broaden our understanding on the underpinnings of RPE signals.\\In this work we propose a study of the formation of prediction error signals in interregional assemblies during reinforcement learning. On this purpose we recorded dual site electrophysiological in-vivo data from VS (including Pallidum) and VTA, during a reversal go/no go task in mice.\\%Details of the task and the data set are presented in \hyperref[sec:Dataset]{~Section \ref*{sec:Dataset}} and \hyperref[sec:MatAndMet]{~Section \ref*{sec:MatAndMet}}.
On this data set we applied a cell assembly detection algorithm (\cite{RussoDurstewitz}). % that will be presented in\hyperref[chap:AssemblyMethod]{~Section \ref*{chap:AssemblyMethod}}.
The novel statistical approach was free to detect spike patterns at any time scale and coordination, enabling so the investigation of the time scales and the inter-units lag activation involved in the detected patterns of spikes. %\\In\hyperref[chap:AssemblyAnalysis]{~Chapter \ref*{chap:AssemblyAnalysis}} we present the cell-assembly analysis intended to understand the nature of VS-VTA direct and indirect pathways.\\%One of the most puzzling question about VS-VTA interactions is if there exists a preferred direction in which the prediction error signal is encoded. 
We focused the analysis on interregional assembly-pairs, namely assembly formed by two neurons, one in VS and the other in VTA, in this way we were able to examine the directionality between VS-VTA interactions through the inter-unit lag activation.\\%Using the algorithm at level of inter-regional pairs the lag in activation between two units, one in VS and the other in VTA, was nothing but the lag in activation between the regions. 
We found that in interregional assembly-pairs VS predominantly led VTA. Moreover interregional assembly-pairs showed a bimodal time scale distribution, such bimodality was solely present in VS-VTA pairs and did not emerge in intraregional pairs.\\We examined the assembly-pair activity patterns of pairs with different time scales and directionalities. It emerged that different time scales and directionalities segregated different activity patterns. Specifically, in more precise time scale, directional assembly-pairs with VTA following VS showed excitatory responses following reward-predicting stimuli, in agreement with prediction error encoding.\\Taking advantage of well defined neuron types classifications both in VS and VTA, we further investigated the specific cell-type composition of the assemblies exhibiting directionality. Interestingly only assembly-pairs formed by putative striatal projection neurons (pSPN) and dopamine neurons (pDAN) were directional in the direction of VS leading VTA.\\Thus, we examined the task related patterns of different assembly-pair types; and we expected that, being directional, SPN-DAN assembly-pairs could exhibit RPE response.\\Indeed, different assembly-pairs types showed different activity patterns in response to external potentially rewarded stimuli. The segregation reflected different encoding features in different assembly-pair types.\\In particular, SPN-DAN assembly-pairs were mainly activated by the rewarded stimulus, and only a small fraction ($\sim12\%$) was activated by both stimuli. The activation started few hundred milliseconds after the stimulus onset and remained high for few hundred milliseconds ([100,400]). SPN-DAN activity pattern suggested that those pairs conveyed the valuation component of RPE signals (\cite{Tobler2003}, \cite{Nomoto2010}, \cite{Schultz2016}). Conversely, FSN-DAN assembly-pairs responded indistinctly to both stimuli, either showing a brief and phasic activation at the stimulus onset or being inhibited by one or both stimuli; these signals suggested that FSN-DAN pairs were involved in motivational and/or hedonic signals. Hence, we putted forth the concept that the assembly-pairs specialize in different aspects of the learning-related coding.\\%\\Detailed discussion will be presented in\hyperref[sec:TaskResp]{~Section \ref*{sec:TaskResp}} and\hyperref[sec:FalseAlCorrRej]{~Section \ref*{sec:FalseAlCorrRej}}.\\
So far the description presented barely considered the dynamic of the learning process. In fact, reward prediction coding signals in dopamine neurons varied according to the probability to get the reward, which was often related to the uncertainty (\cite{Schultz1992}). Dopamine neurons in VTA as well as neurons in VS modified their activity in function to the difference between received and expected outcomes (\cite{Fiorillo}). In similar way, far from being static, the assembly-pair activity modified itself trial by trial, reflecting the dynamic of the learning.\\This dynamic could not be replicated by the study of activity patterns, for this reason we modeled a reinforcement learning model. Crucial terms of reinforcement learning are the uncertainty of the animal to get the reward and the reward prediction error; the first is high when the animal is unsure about the outcome, and decreases as the animal becomes expert; the latter term reflects the ability of the animal to predict the reward: as the animal learnt is supposed to be able to predict the future outcomes. Both the aforementioned terms were modeled as time evolving components, in such a way that we could take into account the fact that, during the task, the animal had to assign and re-assign new value to the presented stimulus.\\Based on broad knowledge, reward prediction error signals are thought to be anti-correlated with the uncertainty of the animal to get the reward and correlated with the prediction error term of the Rescorla-Wagner models.\\Thus, if SPN-DAN assembly-pairs specifically encoded prediction error signals, we expected their activity to anti-correlate with the modeled uncertainty ($\alpha$) and to correlate with the modeled prediction error ($\delta$).\\We modeled two linear Poisson regressions and we regressed the assembly-pairs activity on $\alpha$ and $\delta$. Our SPN-DAN assembly-pairs indeed anti-correlated with the uncertainty and correlated with the prediction error term. Furthermore we noted that such correlations were not found in other assembly-pair types, from which we could conclude that SPN-DAN assembly-pairs specifically conveyed reward prediction signals.  