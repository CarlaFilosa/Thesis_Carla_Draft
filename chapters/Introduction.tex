\chapter{Introduction}
\label{chap:Introduction}
%\section{Motivation}
%\label{sec:Motivation}
The mesolimbic dopamine pathway comprising the ventral tegmental area and projection terminals in the striatum has been identified as a critical neural system involved in processing both the rewarding and aversive behavioral effects of rewarded and unrewarded stimuli.\\Neuronal models of reinforcement learning assume interactions of midbrain dopaminergic neurons and striatum to compute the differences between anticipated and received outcomes (\cite{Schultz2001}, \cite{Schultz2002}, \cite{Fiorillo}, \cite{Eshel1}, \cite{Pagnoni}, \cite{Radua}, \cite{Takahashi2016}). The nature of this cross-areal interaction is however not fully understood.\\In this work we propose an in depth study on the formation of prediction error signals in interregional assemblies during reinforcement learning. On this purpose we record dual site simultaneous electrophysiological in-vivo data from ventral striatum, including Pallidum, (VS) and ventral tegmental area (VTA). Details of the task and the data set are presented in \hyperref[sec:Dataset]{~Section \ref*{sec:Dataset}} and \hyperref[sec:MatAndMet]{~Section \ref*{sec:MatAndMet}}. On this data set we apply a cell assembly detection algorithm (\cite{RussoDurstewitz}), that will be presented in\hyperref[chap:AssemblyMethod]{~Section \ref*{chap:AssemblyMethod}}.
The novel statistical approach is free to detect spike patterns at any time scale and coordination enabling so the investigation of the time scales and the inter-units lag activation involved in those interregional interactions.\\In\hyperref[chap:AssemblyAnalysis]{~Chapter \ref*{chap:AssemblyAnalysis}} we present the cell-assembly analysis intended to understand the nature of VS-VTA direct and indirect pathways.\\One of the most puzzling question about VS-VTA interactions is if there exists a preferred direction in which the prediction error signal is encoded. Restricting the analysis of assemblies at level of assembly-pairs we investigate the directionality between VS-VTA interactions through the inter-unit lag activation. Using the algorithm at level of inter-regional pairs the lag in activation between two units, one in VS and the other in VTA, is nothing but the lag in activation between the regions. Our nomenclature is such that a positive lag means that VS is prior in activation, a negative lag, vice-versa, means that VTA is preceding the activation of VS. VS predominantly leads VTA in detected assembly-pairs.\\Moreover interregional assembly-pairs show a bimodal time scale distribution, such bimodality is solely present in VS-VTA pairs and does not emerge in intraregional pairs. Looking at the assembly-pairs activity it emerges that different time scales and directionalities dissect different activity patterns, in particular directional assembly-pairs with VTA following VS in shorter time scale show activity patterns in agreement with prediction error encoding (\cite{Tobler2003}, \cite{Nomoto2010}, \cite{Schultz2016}).\\Taking advantage of well defined neuron typologies classifications both in VS and VTA, we further investigate the specific cell-type composition of the assemblies exhibiting directionality. Interestingly only assembly-pairs formed by putative striatal projection neurons (pSPN) and dopamine neurons (pDAN) are directional in the direction of VS leading VTA.\\Thus, considering the dissection in different activity patterns brought by different time scales and directionality, we assume that different assembly-pair types are diversified in task related activity. We examine the task related patterns of different assembly-pair types in three windows of interest, defined from crucial moments in the task proposed, the stimulus and the reward. Significant assembly-pairs activity will be assessed by Friedman test.\\The difference in assembly-pairs types task related patterns reflect different coding and encoding features. Based on broad evidence (\cite{Eshel}) dopamine neurons share common response function for reward prediction to guarantee robust information coding inasmuch each dopamine neuron contributes fully to the reward prediction error. However is not proved whether different neuronal interactions involving dopamine neurons specialize in different aspect of reward prediction error. In the last years it has been shown indeed that reward prediction is consisting of different components (\cite{Nomoto2010}, \cite{Fiorillo2013b}, \cite{Schultz2016}). The first, detection component, reflects a unspecific response to stimulus, in order to detect it regardless its association with the reward. After this phase the stimulus is identified in order to assign to it a value, allowing the animal to predict the reward. This second component of prediction error, also called the main component, constitutes biological implementations of the crucial error term for reinforcement learning according to the Reinforcement Learning models. The hypothesis we will make is that the main component signal is formed specifically by SPN-DAN pairs, whilst FSN-DAN pairs could be involved in motivational salience at the odor onset, reflected in the detection component of prediction.\\We build this hypothesis starting from assembly-pair types task related patterns, precisely. SPN-DAN assembly-pairs response show resemblance to reward prediction error signals, that is not equally found in other assembly-pair types.\\The task-related patterns give us strong indications about the possible different assembly-types coding features. From the evolution in trials of the assembly-pairs response, it is clear that the assembly get modified by the high dynamic of task, proper of the learning process. This dynamic cannot be replicated by the study of activity patterns, for this reason 