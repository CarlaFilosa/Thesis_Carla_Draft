\chapter{Introduction}
\label{chap:Introduction}
%\section{Motivation}   
%\label{sec:Motivation}
The mesolimbic dopamine pathway comprising the ventral tegmental area (VTA) and projection terminals in the ventral striatum (VS) has been identified as a critical neural system involved in processing both the rewarding and aversive behavioral effects of rewarded and unrewarded stimuli. In these functions the brain performs simple arithmetic, it compares the expected and the received outcomes and computes the differences between the two, i.e. it computes the reward prediction error (RPE) (\cite{Schultz2001}, \cite{Schultz2002}, \cite{Fiorillo}, \cite{Eshel1}, \cite{Pagnoni}, \cite{Radua}, \cite{Takahashi2016}, \cite{TianHuang}). However, how these computations are implemented in VS-VTA neural circuits remains elusive.
%Previous modeling and experimental studies have shown that simple arithmetic computations may arise from a wealth of nonlinear mechanisms to transform synaptic inputs into output firing at the level of single neurons ( Chance et al., 2002; Holt and Koch, 1997; Silver, 2010 ). However, it remains unknown whether these mechanisms underlie brain computations in a natural, behavioral context.\\
%The VTA receives indeed inputs from many different brain regions and cell-types, but what each of these inputs contributes to RPE computation is unclear. 
%Recently Watabe-Uchida and colleagues have shown that these input neurons show a wide range of different response profiles to cues and rewards (\cite{TianHuang})
Thus, understanding VS-VTA related circuits, will broaden our understanding on the underpinnings of RPE signals.\\In this work we propose a study of the formation of prediction error signals in interregional assemblies during reinforcement learning. On this purpose we recorded dual site electrophysiological in-vivo data from VS (including Pallidum) and VTA, during a reversal go/no go task in mice.\\%Details of the task and the data set are presented in \hyperref[sec:Dataset]{~Section \ref*{sec:Dataset}} and \hyperref[sec:MatAndMet]{~Section \ref*{sec:MatAndMet}}.
On this data set we applied a cell assembly detection algorithm (\cite{RussoDurstewitz}). % that will be presented in\hyperref[chap:AssemblyMethod]{~Section \ref*{chap:AssemblyMethod}}.
The novel statistical approach was free to detect spike patterns at any time scale and coordination, enabling so the investigation of the time scales and the inter-units lag activation involved in the detected patterns of spikes. %\\In\hyperref[chap:AssemblyAnalysis]{~Chapter \ref*{chap:AssemblyAnalysis}} we present the cell-assembly analysis intended to understand the nature of VS-VTA direct and indirect pathways.\\%One of the most puzzling question about VS-VTA interactions is if there exists a preferred direction in which the prediction error signal is encoded. 
We focused the analysis on interregional assembly-pairs, namely assembly formed by two neurons, one in VS and the other in VTA, in this way we were able to examine the directionality between VS-VTA interactions through the inter-unit lag activation.\\%Using the algorithm at level of inter-regional pairs the lag in activation between two units, one in VS and the other in VTA, was nothing but the lag in activation between the regions. 
We found that in interregional assembly-pairs VS predominantly led VTA. Moreover interregional assembly-pairs showed a bimodal time scale distribution, such bimodality was solely present in VS-VTA pairs and did not emerge in intraregional pairs.\\We examined the assembly-pair activity patterns of pairs with different time scales and directionalities. It emerged that different time scales and directionalities separated out different activity patterns, in particular directional assembly-pairs with VTA following VS in shorter time scale showed activity patterns in agreement with prediction error encoding (\cite{Tobler2003}, \cite{Nomoto2010}, \cite{Schultz2016}).\\Taking advantage of well defined neuron types classifications both in VS and VTA, we further investigated the specific cell-type composition of the assemblies exhibiting directionality. Interestingly only assembly-pairs formed by putative striatal projection neurons (pSPN) and dopamine neurons (pDAN) were directional in the direction of VS leading VTA.\\Thus, considering the segregation in different activity patterns brought by different time scales and directionalities, we assumed that different assembly-pair types were diverse in task related activity.\\We examined the task related patterns of different assembly-pair types. Significant assembly-pairs were assessed by the Friedman test. Different assembly-pairs types showed different activity patterns in response to external potentially rewarded stimuli. This segregation reflected different encoding features in different assembly-pair types.\\We putted forth the concept that the assembly-pairs specialize in different aspects of the learning-related coding. In particular we assumed from the activity patterns of SPN-DAN pairs that those pairs types conveyed the RPE signal, whereas FSN-DAN pairs showed motivational or hedonic salience.\\%\\Detailed discussion will be presented in\hyperref[sec:TaskResp]{~Section \ref*{sec:TaskResp}} and\hyperref[sec:FalseAlCorrRej]{~Section \ref*{sec:FalseAlCorrRej}}.\\
From the evolution in trials of the assembly-pairs response, it was clear that the assembly got modified by the high dynamic of task, proper of the learning process. This dynamic could not be replicated by the study of activity patterns, for this reason we model a reinforcement model. Crucial terms of the reinforcement learning were the uncertainty of the animal to get the reward, and the reward prediction error, those terms evolved during the task as the animal learnt and became more experience. In such a way we could take into account that during the task the animal had to assign and re-assign new value to the presented stimulus.\\Reward prediction coding signals in dopamine neurons varied according to the probability to get the reward, which was often related to the uncertainty (\cite{Schultz1992}). Dopamine neurons in VTA as well as neurons in VS modified their activity in function to the difference between received and expected outcomes (\cite{Fiorillo}).\\Based on this knowledge, we assumed that reward prediction error signals anti-correlate with the uncertainty of the animal to get the reward and correlate with prediction error.\\Thus, if SPN-DAN assembly-pairs specifically encoded prediction error signals, we expected their activity to anti-correlate with the uncertainty term of the model and to correlate with the prediction error.\\We modeled two linear Poisson regressions and we regressed out the uncertainty and the prediction error. Our SPN-DAN assembly-pairs indeed anti-correlated with the uncertainty and correlated with the prediction error term, furthermore such correlations were not found in other assembly-pair types, from which we could conclude that SPN-DAN assembly-pairs specifically convey reward prediction signals.  