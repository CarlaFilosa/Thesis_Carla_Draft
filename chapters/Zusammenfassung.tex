\section*{Zusammenfassung}
Ziel vieler Verhalten ist es, das Maß an Belohnung, das erzielt werden kann, zu optimieren. Hierzu muss das Individuum mögliche Vorhersagen über den Zusammenhang zwischen Reizen und Belohnung machen.  Bei diesen Vorhersagen treten natürlich Fehler auf. Vorhersagefehler für Belohnung beschreiben die Differenz zwischen erwarteter und erhaltener Belohnung. Diese Vorhersagefehler sind im Hirn in der Aktivität dopaminerger Mittelhirnneurone in der VTA präsentiert und werden durch belohnte Stimuli und Belohnung hervorgerufen. Diese Dopaminsignale sind essentiell für Lernvorgänge. Insbesondere Modelle des Verstärkungslernens lassen Interaktionen zwischen dopaminergen Mittelhirnneuronen und dem ventralen Striatum (VS) vermuten, die die Differenz zwischen antizipierter und erhaltener Belohnung kodieren. Diese Interaktion konnte bisher aber nicht direkt gezeigt werden.\\In dieser Arbeit untersuchte ich die Entstehung des Vorhersagefehlers in VS-VTA Ensembles während des Verstärkungslernens. Diese Arbeit basiert auf in vivo elektrophysiologischen Untersuchungen im VS -inklusive dem ventralen Pallidum (VP)- und der VTA während einer go/no-go Lernaufgabe in Mäusen. Ich untersuchte insbesondere die Interaktion zwischen VP Neuronen und dopaminergen Neuronen und die zwischen striatalen Projektionsneuronen und dopaminergen Zellen.\\\\Die Analyse der Ensembleaktivität erfolgte mit einem neuen Detektionsalgorithmus. Es zeigte sich hier, dass speziell striatale Projektionsneurone in den Ensembles vor den dopaminergen Mittelhirnneuronen aktiv sind, diese also informieren.\\\\Die Analyse der Ensemble-Aktivität während der Belohnungslernens zeigte ferner, dass Ensembles zwischen striatalen Projektionsneuronen und dopaminergen Neuronen selektiv durch den belohnten Stimulus aktiviert wurden. Demgegenüber zeigten Ensembles mit VP Neuronen eine unspezifische Stimulusaktivierung, was auf eine generelle Kodierung von Stimulussalienz hinweist.\\\\Diese differentielle Aktivierung verschiedener Ensembletypen weißt auf spezifische Funktionen beim Lernen hin. Ich testete diese Frage mit einem Model für Verstärkungslernen basierend auf Rescorla-Wagner Lernregeln, die den Vorhersagefehler und die Unsicherheit der eigentlichen Vorhersage abbilden. In der Tat korrelierte die Aktivität von Ensembles zwischen striatalen Projektionsneuronen und dopaminergen Neuronen mit dem Vorhersagefehler und der Unsicherheit der eigentlichen Vorhersage. Dies trifft nicht auf Ensembles zu, an denen VP Neurone mit dopaminergen Neuronen interagierten. Zusammenfassend liefern diese Daten direkte Hinweise, dass das ventrale Striatum die VTA über den Vorhersagefehler informiert, der in den interarealen Ensembles zwischen beiden Arealen repräsentiert ist.
