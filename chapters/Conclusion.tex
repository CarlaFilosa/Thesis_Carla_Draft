\chapter{Conclusion and discussion}
\label{chap:Conclusion}
Neuronal models of reinforcement learning assume interactions of midbrain dopaminergic neurons in ventral tegmental area (VTA) and ventral striatum (VS) to compute the differences between anticipated and received outcomes, i.e. they compute the reward prediction error (RPE).\\We proposed here a study of the formation of RPE signals in VS-VTA interregional assemblies during learning to broaden our knowledge of the underpinnings of such signals. From this study we have drawn the following conclusion:
\begin{itemize}
    \item \textbf{Assembly-pairs occurrence, time scales and directionality:}\\We recorded neuronal activity in VS, including pallidum (VP), and VTA during a reversal learning go-no task in mice. In VS striatal projection neurons (SPN) and fast spiking neurons (FSN) were the more frequent cells types. In VTA we recorded a good fraction of gabaergic (GABA) and dopamine neurons (DAN). In this data set we further applied an unsupervised cell-assemblies detection method, to detect assemblies of synchronously ($lag=0$) and sequentially ($lag\neq0$) active units at arbitrary time scales ($\Delta$).\\The $\chi^2$ test revealed that SPN and DAN had high tendency to agglomerate in assembly together, as well as FSN and GABA.\\While we observed assemblies of temporal precision at the scale of few tens of milliseconds only within either VS or VTA, assemblies of lower temporal precision were detected across VS-VTA units. The temporal precision of this last group displayed a bimodal distribution with peaks around hundred milliseconds and one second.\\We focused on the more precise temporal scale first because these times did not interfere with the characteristic temporal scales of the experiment, such as the odor duration; second because the reward prediction error signals typically involve fast temporal scales (few hundred milliseconds). Interestingly the lags of more temporally precise assemblies displayed an asymmetric distribution indicating VS leading VTA.\\ Specifically, these directional assemblies were composed of SPN and DAN neurons. This was a functional directionality, the units in assemblies did not have to be directly connected by a synapse. Indeed, the SPN-DAN communication could occur both directly and relayed through interspersed neurons.\\In summary, we revealed here the multiple time scale and functional network of the striatum mid-brain interaction during a reversal assignment task.  
    \item \textbf{Assembly-pairs activity patterns:}\\
    Assemblies with different time scales and directionalities segregated different task related activity patterns.\\A large fraction of SPN-DAN assembly-pairs became selectively active at the rewarded stimulus onset, whereas a small fraction were active at the time of the reward retrieval. This kind of activation resembled the stereotypical RPE signals. Indeed RPE signal is formed after few hundred milliseconds from the stimulus onset (\cite{Tobler2003}, \cite{Nomoto2010}, \cite{Schultz2016}).\\FSN-DAN pairs were unselectively activated by rewarded or unrewarded stimuli. Furthermore their activation was diverse among different assembly-pairs, thereby suggesting that those assembly-pair types could be involved in motivational or hedonic signals rather than in RPE signals.\\These results different assembly-pair types with DAN had different reward-related coding features. In particular we putted forth the concept that SPN-DAN specialized in the valuation of component RPE. This component constitutes biological implementations of the crucial error term for reinforcement learning according to the Rescorla–Wagner-like models and temporal difference reinforcement models; such a signal is appropriate for mediating learning and updating of reward predictions for approach behavior and economic decisions. Probably the most important potential that utility has for neuroscience lies in the assumption that utility provides an internal, private metric for subjective reward value. Utility as an internal value reflects individual choice preferences and constitutes a mathematical function of objective, physical reward amount. 
    \item \textbf{Correlation with reinforcement learning model function:}\\
\end{itemize}
  These findings provide a network coding mechanism of how dopaminergic learning signals promote value assignment to generate an assembly prediction code to dopaminergic midbrain neurons. 