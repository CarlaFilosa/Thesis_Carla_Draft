\chapter{Conclusion and discussion}
\label{chap:Conclusion}
Neuronal models of reinforcement learning assume interactions of midbrain dopamine neurons in ventral tegmental area (VTA) and ventral striatum (VS) to compute the differences between anticipated and received outcomes. These signals define the reward prediction error (RPE), essential in learning to predict the future and maximise the reward. However how RPE signals are formed and encoded in VS-VTA circuit remains elusive.\\We studied the formation of (RPE) signals in VS-VTA interregional assemblies during learning to broaden our knowledge of the underpinnings of such signals. Here I discuss advantages and limitations the applied techniques in order to draw final conclusion.
\section{Methodological discussion}
\subsection{The cell-assembly detection method (CAD)}
CAD, presented in \hyperref[chap:AssemblyMethod]{~Chapter\ref*{chap:AssemblyMethod}}, is particularly suited to investigate neural VS-VTA interaction involved in basal ganglia and VTA pathways in dynamic tasks such as reinforcement learning. Indeed, reversal learning go/no go tasks are highly dynamic and such learning dynamic implies quick changes in network states, that can be taken into account by CAD with the correction for non-stationarity. Moreover whilst other methods often propose to detect cell assemblies by aligning the population activity to task/behavioral events, CAD sums evidence of assembly activity throughout the task and session. This makes possible the detection of the same assembly even if its time of activation changes within the trial.\\A direct comparison on ground truth non-stationarity data between CAD and PCA-based methods, revealed how the correction for non-stationarity is critical in the detection performance. Indeed, the presence of non-stationarities massively reduced the performance of PCA-based methods, both reducing the assembly detection rate and increasing the false discovery rate.\\Furthemore CAD, in contrast to other recent methods, can establish the identity of the units composing a cell assembly; this property was in fact crucial in the study of VS-VTA interactions, allowing to assess different coding features for different assembly-pair types.
CAD: comparison with other methods.
\begin{itemize}
    \item A direct comparison on ground truth non-stationarity data between CAD and PCA-based methods, revealed how the correction for non-stationarity is critical in the detection performance. Indeed, the presence of non-stationarities massively reduced the performance of PCA-based methods, both reducing the assembly detection rate and increasing the false discovery rate.
\item In a recent paper by Fukai group, a new algorithm to detect cell assemblies is proposed43. The algorithm is based on edit similarity, a measure to compute the distance between strings, and on two clustering steps, to establish windows of similar spike profiles repeating during the recordings. In the paper, a direct comparison with CAD is made. While the methods, in general, when accounting for both correct detection and false discovery rate, performed roughly similarly, the method by Watanabe et al. (2019), in contrast to CAD, does not perform significance testing for identified clusters, can’t establish the identity of the units composing a cell assembly, finds assemblies with patterns of only a fixed length arbitrarily set by the experimenter, and relies on the fine- tuning of a number of parameters.
%\item In a review published by the Grün lab57, nine different detection methods for spike patterns were compared on the basis of the ability to identify the assembly composing units, the null hypothesis tested for assembly detection, and the type of detectable assembly patterns. CAD and SPADE, a method based on frequent itemset mining33,37, were found to be the most complete methodologies in the sense that both, in contrast to others, could account for arbitrary activity patterns and detect the identity of the units composing the assemblies (the efficiency of the methods wasn’t tested). It is, however, important to note that, while CAD is able to detect assemblies at any temporal resolution, SPADE is based on the analysis of binary time series and can, therefore, detect exclusively spike patterns with a sharp temporal resolution. Moreover, because of its high
%computational load, SPADE – in contrast to CAD - can be applied only on (few minutes) short segments of data.
%Finally, while one or the other feature of CAD is shared with one or the other existing methods for assembly detection, none of them, to the best of my knowledge, is able to detect such wide range of assembly structures. In particular, none of the other available methods explicitly addresses the temporal resolution at which assemblies coordinate. Such temporal resolution is, however, a core feature of the assembly coding and it can vary both depending on the cortical region and the cognitive function the region is required to perform by the specific task in place54.
\end{itemize}

%Beyond CAD: excitatory-inhibitory assemblies. Typically, assembly mining methods, CAD included, aim to detect patterns of coordinated increase in spikes/firing rate. Therefore, if a recurrent pattern is composed of both an increase and a decrease in rate of two sets of units, only the excited set would be detected as an assembly. It is known that excitatory-inhibitory balance regulates both spontaneous and sensory-evoked responses of many brain regions58. A particular interest has been recently devoted to the coding role of inhibition59. Inhibitory engrams7 or shadows patterns60, the negative counterpart of cell assembly activation, are thought to induce behavioral habituation61, increase the storage capacity of a network60,62, increase the dimensionality of the population code59 and, by targeted disinhibition, facilitate the recall of latent memories63. To the present date, however, very few machine learning methodologies are available for mining coordinated inhibitory-excitatory patterns in electrophysiological recordings. Billeh et al.35 suggest to detect cell assemblies by applying the Markov stability method to the functional connectivity matrix estimated on the parallel spike time series. Excitatory and inhibitory interactions are accounted for by the functional connectivity matrix by means of two different metrics. The authors, however, suggest the exclusive use of the excitatory metric whenever the cell typology is not confirmed by conditional genetics. Since the vast majority of experimental dataset lack this kind of information, the method has limited applicability. Moreover, inhibitory coordination can also occur between two excitatory neurons when, for example, they are part of competing attractors (Russo at al. 2008, Russo and Treves 2012)64,65. With a different approach, Li et al.39 suggest assigning a positive/negative surprisals code to the time bins where the ongoing inter-spike-interval of a unit is significantly smaller/larger then what expected by the gamma distribution estimated on its inter-spike-interval distribution. Surprisal patterns, i.e. cell assemblies, are then extracted through independent component analysis. The method, which does account for excitatory inhibitory patterns, suffers, however, of a number of limitations: the detection of positive/negative surprisals events by means of a parametric gamma distribution may lead to false discoveries if not corrected, as in this case, for non-stationarities; the number of assemblies (blind sources) is arbitrarily chosen; only synchronous patterns are detectable; and it can only be applied on spike data. Finally, it is to be mentioned the work of Gärtner et al.66 for the detection of joint pausiness in parallel spike trains. Yet, the latter method accounts exclusively for joined reductions, and not excesses, in neuronal discharge. In conclusion, to the best of my knowledge, no satisfactory algorithm for the detection and test of joined excitatory and inhibitory patterns is currently available.
\subsection{The reinforcement learning model (Q L-F)}


\begin{itemize}
    \item \textbf{Assembly-pairs occurrence, time scales and directionality:}\\Neuronal activity in VS, including ventral pallidum (VP), and VTA was recorded during a reversal learning go-no task in mice. In VS/VP striatal projection neurons (SPN) and fast spiking neurons (FSN) were the more frequent cells types. In VTA a good fraction of dopamine neurons (DAN) and  gabaergic (GABA) neurons were recorded. In this data set I further applied an unsupervised cell-assemblies detection method, to detect assemblies of synchronously ($lag=0$) and sequentially ($lag\neq0$) active units at arbitrary time scales ($\Delta$).\\Using $\chi^2$ test, I assessed whether different cell-types had different tendency to agglomerate together; that revealed that two groups of units had high tendency to agglomerate in assembly, namely SPN and DAN together, and FSN and GABA together.\\While I observed assemblies of temporal precision at the scale of few tens of milliseconds only within either VS or VTA, assemblies of lower temporal precision were detected across VS-VTA units. The temporal precision of this last group displayed a bimodal distribution with peaks around hundred milliseconds and one second.\\I focused on the more precise temporal scale first not to interfere with the characteristic temporal scales of the experiment, such as the odor duration; second because the reward prediction error signals typically involve fast temporal scales (few hundred milliseconds). Interestingly the lags of more temporally precise assemblies displayed an asymmetric distribution indicating VS leading VTA.\\ Specifically, these directional assemblies were composed of SPN and DAN neurons. The directionality was functional: the units in assemblies did not have to be directly connected by a synapse. Indeed, the SPN-DAN communication could occur both directly and relayed through interspersed neurons.\\In summary, I revealed here the multiple time scale and functional network of the striatum mid-brain interaction during a reversal assignment task.  
    \item \textbf{Assembly-pairs activity patterns:}\\
    Assemblies with different time scales and directionalities segregated different task related activity patterns.\\A large fraction of SPN-DAN assembly-pairs became selectively active at the rewarded stimulus onset and remained active for few hundred millisecond ([100, 400] ms); whereas a small fraction were active at the time of the reward retrieval. This kind of activation resembled the stereotypical RPE signals. Indeed RPE signal is formed after few hundred milliseconds from the stimulus onset. Usually preceded by the phasic activation of the stimulus detection, the RPE signal evolves then in a broader activation, to value the stimulus and predict the reward (\cite{Tobler2003}, \cite{Nomoto2010}, \cite{Schultz2016}).\\%Thus, I assumed SPN-DAN assembly-pairs conveying the valuation component of RPE signals.\\
    FSN-DAN pairs, conversely, were unselectively and phasic activated by rewarded or unrewarded stimuli. Furthermore their activation was diverse among different assembly-pairs, thereby suggesting that those assembly-pair types could be involved in motivational or hedonic signals, rather than in RPE signals.\\These results suggested that different assembly-pair types with DAN had different reward-related coding features. I putted forth the concept that SPN-DAN specifically conveyed the valuation of component RPE. This component constitutes biological implementations of the crucial error term for reinforcement learning according to the Rescorla–Wagner-like models. 
    \item \textbf{Correlation with reinforcement learning model function:}\\The activity patterns gave us information about the average activity across trials in different moments of the trial, by considering how the signal evolved within the trial; however this view lacked the trial by trial evolution. Nevertheless, the assembly-pair activity got modified by the learning process, as the animal, in base of its comprehension of the task rule, dynamically adapted the value assigned to stimulus.\\Reinforcement learning models capture such dynamic, parameterizing the learning functions. I proposed here a learning-forgetting model, and I focused on the crucial terms for RPE signals, namely the uncertainty to get the reward and the prediction error. The prediction error, called $\delta$ in the model, was nothing but the mathematical difference between the expected reward value and the actual reward; the uncertainty, $\alpha$, was a time-dependent component, modulated by recent predictions and the prediction error $\delta$. This component mimicked the uncertainty of the animal to get the reward, and its value was high at the beginning of the task and decreased ad the animal learnt the rule, to raise up again at the beginning of the reversal phase.\\Conversely, RPE signals in dopamine neurons are such that neuronal activity increases monotonically after the stimulus onset with the probability to get the reward, as consequence of the increase of the certainty of the animal to get the reward when it learnt the task; meanwhile at the reward delivery time, the neural activity decreases as the ability of the animal to predict the reward increases (see \hyperref[chap:Overview]{~Chapter \ref*{chap:Overview}}). In other words, during learning the peak of neural activity shifts back from the reward retrieval time to the stimulus onset.\\Based on this knowledge I assumed that, if an assembly-pair conveyed RPE signals, its activity anti-correlated with the uncertainty in the stimulus window, and correlated with the prediction error in the reward window.\\This was assessed with two Poisson linear regression of the assembly-pairs activity on the uncertainty ($\alpha$) in the CS window and on the prediction error ($\delta$) in the US window. The regression coefficients distributions confirmed that SPN-DAN assembly-pairs specifically conveyed RPE signals; this signals were not found in FSN-DAN assembly-pairs, that could instead represent motivational salience or hedonic signals. In conclusion in this thesis was provided a specific coding mechanism of how assignment value and reward prediction error signals are formed and encoded in VS-VTA assemblies. 
\end{itemize}
  