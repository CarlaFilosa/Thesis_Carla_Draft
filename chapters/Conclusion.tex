\chapter{Conclusion and discussion}
\label{chap:Conclusion}
Neuronal models of reinforcement learning assume interactions of midbrain dopamine neurons in ventral tegmental area (VTA) and ventral striatum (VS) to compute the differences between anticipated and received outcomes. These signals define the reward prediction error (RPE), essential in learning to predict the future and maximise the reward. However how RPE signals are formed and encoded in VS-VTA circuit had remained elusive (\cite{Schultz2016}).\\I studied the formation of RPE signals in VS-VTA interregional assemblies during learning to broaden the knowledge of the underpinnings of such signals. Here I discuss advantages and limitations the applied techniques in order to draw final conclusion.
\subsubsection{Methodological discussion on cell-assembly detection method (CAD)}
I employed the cell-assembly detection method (CAD) used for the analysis, developed by \citeay{RussoDurstewitz}. The term cell-assembly was first used by Hebb to describe a network of neurons that is being activated repeatedly during a certain mental process (\cite{Hebb}). This term is used loosely to describe a group of neurons that perform a given action or represent a given percept or concept in the brain. Cell assemblies may vary in the precision of  the temporal coordination of their units, generating patterns with  temporal coordination ranging from  the  millisecond precision to broader rate modulations. CAD is able to detect assemblies of synchronously and sequentially active units at arbitrary time scales, enabling so the investigation of the time scales and the inter-units lag activation involved in the detected patterns of spikes. In particular, none of the other available methods explicitly addresses the temporal resolution at which assemblies coordinate (\cite{Gruen}, \cite{Tavoni}, \cite{Billeh}). Such temporal resolution is, however, a core feature of the assembly coding and it can vary depending on the brain region and the cognitive function that is performed in a specific task.\\CAD is suited to investigate neural VS-VTA interaction involved in basal ganglia and VTA pathways in dynamic tasks such as reinforcement learning. Indeed, reversal learning go/no-go tasks are highly dynamic and such learning dynamic implies quick changes in network states, that can be taken into account by CAD with the correction for non-stationarity. Moreover whilst other methods often propose to detect cell assemblies by aligning the population activity to task/behavioral events, CAD sums evidence of assembly activity throughout the task and session. This allows the detection of the same assembly even if its time of activation changes within the trial.\\A direct comparison on ground truth non-stationarity data between CAD and PCA-based methods (\cite{Lopes}, \cite{RussoDurstewitz}), revealed how the correction for non-stationarity is critical in the detection performance. Indeed, the presence of non-stationarities massively reduced the performance of PCA-based methods, both reducing the assembly detection rate and increasing the false discovery rate.\\Furthermore CAD, in contrast to other recent methods (\cite{Watanabe2019}), can establish the identity of the units composing a cell assembly; this property was in fact crucial for the study of VS-VTA interactions, allowing to assess different coding features for different assembly-pair types.
\subsubsection{Limitations of CAD}
CAD aims to detect patterns of coordinated increase in spikes/firing rate. Therefore, if a recurrent pattern is composed of both an increase and a decrease in rate of two sets of units, only the excited set would be detected as an assembly. However excitatory-inhibitory balance regulates both spontaneous and sensory-evoked responses of many brain regions and pathways, including the mesolimbic pathway. This means that in detected assembly-pairs including leading inhibitory units, like directional SPN-DAN assembly-pairs, the DAN activation following the SPN activation was not caused by a synapse, but was the consequence of a functional interaction that most probably includes interspersed units. However no satisfactory algorithm for the detection and test of joined excitatory and inhibitory patterns is currently available.
\subsubsection{Discussion on the reinforcement learning model}
Further I used the reinforcement learning model (Q L-F) to parameterize the learning functions, on which later I regressed the assembly-pair activity. Q L-F was a hybrid Rescorla-Wagner model with Pearce-Hall update mechanism (\cite{Koppe}, \cite{Li}, \cite{Costa}) and a time dependent forgetting component, which accounted for the update of the unchosen option. Q L-F was compared with other three models and resulted to be the best model according to likelihood-ratio test (LRT), used the cases in which the models to compare were nested; and to the Bayesian information criterion (BIC), used in cases in which the models to compare had the same number of parameters.\\In Q L-F two actions $"$lick$"$/$"$no-lick$"$ and two states CS+/- corresponding to the rewarded/unrewarded stimulus were implemented. In this way, the lick was considered an instrumental action, rather than an impulsive action; the task presented was indeed such that the animal had to learn to lick at the rewarded odor presentation and to not lick at the unrewarded odor; only when both part of the task were well performed the animal  reached the performance criterion. A recent work (\cite{SchultzMot}) showed that in go/no-go tasks the performance reflects balance between impulsive and instrumental components of behavior; in particular behavior is decomposed in two  responses: a primary lick response, defined by a sharp early peak immediately after stimulus onset, which correspond to the impulsive lick, and a broad, multi-peaked secondary response, which corresponds to the instrumental component. I assumed here that the secondary response was driven by the RPE signals. A model that could take into account the shift in behavior from the impulsive to the instrumental component would require a motivational time-component term (\cite{SchultzMot}). Accounting for motivation may have further helped to isolate instrumental components, yet my purpose if this study was to prove  whether the assemblies encoded for RPE components. Therefore a motivation term was not introduced here. However, by modelling a pure Pavlovian model I noticed that relevant results were consistent with Q L-F, although less significant (see \hyperref[chap:SimpRL]{~Appendix \ref*{chap:SimpRL}} for details). 
\subsubsection{Conclusion}
Based on the presented results and the above discussed limitations of the applied techniques, I have drawn the following conclusion:

\subsubsection{Assembly-pairs occurrence, time scales and directionality}
Neuronal activity in VS, including ventral pallidum (VP), and VTA was recorded during a reversal learning go-no task in mice. In VS/VP, striatal projection neurons (SPN) and fast spiking neurons (FSN) were the dominant cell-detected types. In VTA a good fraction of dopamine neurons (DAN) and  gabaergic (GABA) neurons were recorded.\\Assemblies were examined between dominants cell-types of VS and VTA. In this data set I applied an unsupervised cell-assemblies detection method, to detect assemblies of synchronously ($lag=0$) and sequentially ($lag\neq0$) active units at arbitrary time scales ($\Delta$).\\Using $\chi^2$ test, I could show that different cell-types had different tendency to agglomerate together: it revealed that two groups of units had high tendency to agglomerate in assembly, namely SPN and DAN together, and FSN and GABA together.\\While I observed assemblies of temporal precision at the scale of few tens of milliseconds only within VS or VTA, assemblies of lower temporal precision were detected across VS-VTA units. The temporal precision of VS-VTA assemblies displayed a bimodal distribution that peaked around hundred milliseconds and one second. I focused on the more precise temporal scale of VS-VTA assemblies, first not to interfere with the characteristic temporal scales of the experiment, such as the odor duration; second because the reward prediction error signals typically involve fast temporal scales (few hundred milliseconds).\\Interestingly, dividing the assemblies according to the underlying cell-types it emerged that assemblies composed of SPN and DAN neurons were directional on the direction of VS leading VTA. Importantly the directionality is functional: meaning that the units in assemblies do not have to be directly connected by a synapse. Indeed, the SPN-DAN communication may occur both directly and relayed through interspersed neurons (\cite{Ikemoto}).\\In summary, I revealed here the multiple time scale and functional network of the striatum mid-brain interaction during a reversal assignment task.  
   
\subsubsection{Assembly-pairs activity patterns}
Assemblies with different time scales and directionalities segregated different task related activity patterns.\\A large fraction of SPN-DAN assembly-pairs were selectively active at the onset of the rewarded stimulus and remained active for few hundred millisecond ([100, 400] ms); whereas a small fraction was active also at the time of the reward retrieval. Together these patterns of activation resembled the stereotypical RPE signals. Indeed RPE signal is formed after few hundred milliseconds from the stimulus onset. This signal is usually preceded by the phasic activation of the stimulus detection, it then evolves in a broader activation related to the value of the stimulus and prediction of the reward (\cite{Tobler2003}, \cite{Nomoto2010}, \cite{Schultz2016}).\\%Thus, I assumed SPN-DAN assembly-pairs conveying the valuation component of RPE signals.\\
FSN-DAN pairs, conversely, were unselectively activated by rewarded or unrewarded stimuli; thereby suggesting that those assembly-pair types could be involved in hedonic signals, rather than in RPE signals.\\These results suggested that different assembly-pair types with DAN had different reward-related coding features. I put forth the hypothesis that SPN-DAN specifically conveyed the valuation of component RPE.

\subsubsection{Correlation with reinforcement learning model function}
The activity patterns gave us information about the average activity across trials in different moments of the trial, by considering how the signal evolved within the trial; however this view lacked the trial by trial evolution. Nevertheless, the assembly-pair activity got modified by the learning process, as the animal dynamically adapted the value assigned to stimulus.\\Reinforcement learning models capture such dynamic, parameterizing the learning functions. I employed here a learning-forgetting model (Q L-F), and I focused on the crucial terms for RPE signals, namely the uncertainty to get the reward and the prediction error. The prediction error, called $\delta$ in the model, was nothing but the mathematical difference between the expected reward value and the actual reward; the uncertainty, $\alpha$, is a time-dependent component and is modulated by recent predictions and the prediction error $\delta$. $\alpha$ mimicked the uncertainty of the animal to get the reward, and its value was high at the beginning of the task and decreased as the animal learnt the rule, to rise again at the beginning of the reversal phase.\\Conversely, RPE signals in dopamine neurons are such that neuronal activity increases monotonically after the stimulus onset with the probability to get the reward, as consequence of the increase of the certainty of the animal to get the reward when it learnt the task; meanwhile at the reward delivery time, the neural activity decreases as the ability of the animal increases to predict the reward (see \hyperref[chap:Overview]{~Chapter \ref*{chap:Overview}}). In other words, during learning the peak of neural activity shifts back from the reward retrieval time to the stimulus onset.\\Based on this knowledge I assumed that, if an assembly-pair conveyed RPE signals, its activity anti-correlated with the uncertainty in the stimulus window, and correlated with the prediction error in the reward window.\\This was assessed with two Poisson linear regression of the assembly-pairs activity on the uncertainty about the outcome ($\alpha$) in the CS window and on the prediction error ($\delta$) in the US window. The regression coefficients distributions confirmed that SPN-DAN assembly-pairs specifically conveyed RPE signals; these signals were not found in FSN-DAN assembly-pairs, that could instead represent hedonic signals or salience related to the detection of the signal, but not informative about the outcome. In conclusion in this thesis I provided a specific coding mechanism of how assignment value and RPE signals are formed during learning of stimulus-outcome $"$association$"$.

\subsubsection{Biological meaning}
Altered dopamine signaling is involved in many human disorders, from Parkinson$'$s disease to drug addiction, as well as in the formation of social bonds. Understanding how dopamine signals are generated and encoded may be crucial for a better understanding of physiological behavior and human disorders.\\Toward this aim I studied how the RPE signals are formed in VS-VTA assembly-pairs and further I have shown that different assembly-pairs make specific contribution to RPE dopamine signals.\\\\I found that FSN-DAN interactions were not parameterized by the RPE term formalized in Rescorla-Wagner-like models, and, based on their unselective response, I hypothesized that FSN-DAN pairs were involved in hedonic salience, not informative about the specific outcome. This hypothesis would be compatible with the \citeay{Berridge} motivational salience concept, based on neuronal responses in VP.\\\\Conversely I have shown that SPN-DAN interaction specifically encode the main component of RPE signals, that are informative and predictive about the outcome. RPE signals are thought to be the basis of the learning and thus reflect high-level behavior, thus these findings may help to understand dopamine actions both in physiological behaviors and psychiatric disorders. Importantly studies highlighted that numerous behaviors, including choice, confidence, contextual expectations, can be modulated on the basis of  RPE computations (\cite{Gadagkar}, \cite{Stauffer}). Further learnt natural behaviors like the formation of social bonds are presumably related on VS-VTA interactions (\cite{Ungless2004}, \cite{Walum}).\\\\These finding suggested that interactions between VS-VTA convey not-outcome specific signals and outcome selective and predictive signals through different functional pathways. These encoding features together guarantee the full prediction coding.


  