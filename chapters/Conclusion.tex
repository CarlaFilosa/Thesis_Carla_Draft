\chapter{Discussion}
\label{chap:Conclusion}
In the here presented work, I examined the interregional assembly activity between ventral striatum (VS) and ventral tegmental area (VTA) during reinforcement learning. The purpose of this work, was to understand the interaction between the two brain regions and specifically the task-related activity of these assemblies. Towards, to detect assemblies, I employed a recently developed algorithm that I will discuss initially. Based on the assembly detection and characterization, I will then discuss how I investigated whether the assembly activity is related to reward prediction error signaling.
\subsubsection{Methodological discussion on cell-assembly detection method (CAD)}
The nature of cortical representations is central for an understanding of how memory and cognitive operations are implemented by the brain. The scientist Donald Hebb outlined a comprehensive biological theory of psychological function (\cite{Hebb}). His theory relates psychological phenomena as learning, perception, motivation, and emotion to anatomical structures and to physiological processes in the brain. The connection made by Hebb$'$s theory was truly ground breaking when it was published and it defined the program of theoretical neuroscience, that is prevailing to the current day. Hebb built his theory on the definition of a new concept that he called the cell-assembly. A cell assembly refers to a group of neurons organized as a single, functional unit. The stimulation of a constituent neuron results in the activation of the entire group. Under this hypothesis, whenever a subject repeatedly faces a specific cognitive task, the assemblies processing such task will repeatedly activate. However the terms lacks of stringent definition, and it has loosely used to denote anything from the precise zero-phase-lag spike synchronization in a defined subset of neurons (\cite{Abeles}, \cite{Roelfsema}, \cite{Diesmann}, \cite{Harris2003}) to temporally coherent changes in average firing rates on larger time scales (\cite{Goldman}, \cite{Durstewitz}); meaning that cell assemblies may vary in the precision of the temporal coordination of their units, generating patterns with temporal coordination ranging from the millisecond precision to broader rate modulations (\cite{RussoDurstewitz}). The detection of cell assemblies from neural data is based on the idea that each an assembly it reveals as a repetitive activity pattern in the data; and thus it is identifiable as a supra-chance recurrent pattern. However as the number of the recorded units increases the number of assembly patterns to test provokes the combinatorial explosion. Moreover, non-stationarities present in neural data can affect the reliability of the test if they are not account for.\\To overcome some of the challenges outlined above, most techniques limit the search for assembly patterns to one or the other specific time-scale or coordination (\cite{Torre}, \cite{Gruen} \cite{Tavoni}, \cite{Billeh}). Such temporal resolution is, however, a core feature of the assembly coding and it can vary depending on the brain region and the cognitive function that is performed in a specific task.\\In this work I applied a cell assembly-detection method, developed by \citeay{RussoDurstewitz}, which is able to detect assemblies of synchronously and sequentially active units at arbitrary time scales, enabling so the investigation of the time scales and the inter-units lag activation involved in the detected patterns of spikes. CAD is based on a fast parametric test statistic which allows the unsupervised scanning and statistical testing of a large number of configurations. An efficient a-priori algorithm recursively agglomerate units into larger groups allowing the detection of assemblies composed by an arbitrary number of units. By binning the time series at different bin sizes and reducing broader temporal scale processes to the sum of parallel Bernulli sub-processes CAD is able to test assemblies at arbitrary time scales. Finally thanks to a difference statistic the method is resistant to non-stationarity in the time series, avoiding the detection of false positive. Thanks to these methodological advances the method is not only able to detect a large variety of assembly constellations, avoiding the a priori selection of a specific theoretical hypothesis, but infers for each specific dataset the relevant time scales of the information encoding.\\A direct comparison on ground truth non-stationarity data between CAD and PCA-based methods (\cite{Lopes}, \cite{RussoDurstewitz}), revealed how the correction for non-stationarity is critical in the detection performance. Indeed, the presence of non-stationarities massively reduced the performance of PCA-based methods, both reducing the assembly detection rate and increasing the false discovery rate.\\Furthermore CAD, in contrast to other recent methods (\cite{Watanabe2019}), can establish the identity of the units composing a cell assembly; this property was in fact crucial for the study of VS-VTA interactions, allowing to assess different coding features for different assembly-pair types.\\
Finally, assembly mining methods, CAD included, aim typically to detect patterns of coordinated increase in spikes/firing rate. Therefore, if a recurrent pattern is composed of both an increase and a decrease in rate of two sets of units, only the excited set would be detected as an assembly. 

\subsubsection{Interregional assembly activity during reinforcement learning}
\textbf{Nature of detected assemblies.} In line with the above considerations on CAD, the here presented analysis was only sensitive to assembly pairs where both participating units displayed excitatory responses. SPNs are inhibitory projection neurons. This means that in the detected assembly-pairs with leading inhibitory units, like directional SPN-DAN assembly-pairs, the DAN activation following the SPN activation was not necessarily caused by direct monosynaptic connections, but rather as the consequence of a functional interaction that most probably includes interspersed units and disinhibition. Alternatively, as is the case for DAN with rebound burst features, burst firing can occur, after a first transient inhibition, in response to firing of presynaptic inhibitory neurons. 

\subsubsection{Cell-types underlying interregional assemblies} 
Both ventral striatum and VTA contain different cell types. Knowing the cell types that constitute the assemblies can help to interpret their potential functional implications. Also, importantly, one may predict that certain cell types form preferentially assemblies that encode specific task related events. Within the ventral striatum, also comprising the ventral pallidum, we found two cell types based on their baseline firing frequencies. One cell type was striatal projection neurons (SPN) with low firing frequencies (\cite{Kravitz}) and the other one was fast spiking neurons (FSN) compatible with ventral pallidum units (\cite{Heimer1982}). The VTA neurons were classified based on a different approach. Neurons were clustered according to significant activations to reward with transient or persistent activations in the stimulus and reward time window as previously established by Uchida and co-workers (\cite{Uchida}). This classifications was confirmed for DAN by optogenetic tagging in DAT:Cre mice expressing ChR2 selectively in DAN. Based on this classification approach we found two cell types in VTA representing the majority of recorded cells: the first cell-type were DANs and the second were gabaergic (GABA) neurons.\\\\
Assemblies were examined between these dominant cell-types of ventral striatum and VTA. In this data set I applied the CAD method, to detect assemblies of synchronously ($lag=0$) and sequentially ($lag\neq0$) active units at arbitrary time scales. Using $\chi^2$ statistics, I could show that different cell-types preferentially formed functional assemblies. In particular, SPN-DAN and FSN-GABA assemblies showed high tendency to agglomerate in together assemblies. Yet, I also observed a larger group of significant FSN-DAN assemblies. For the further analysis, I focused mainly on two assembly types: SPN-DAN and FSN-DAN. The main motivation for this selection was the focus of this work on reward prediction error signaling where DAN play the central role.\\\\ 
\subsubsection{Temporal organization of assembly activity}  
The detected assemblies had very different temporal precision. Assemblies of temporal precision at the scale of few tens of milliseconds were only detected in intra-regional assembly within either VS or VTA, while assemblies of lower temporal precision were detected across VS-VTA units. The temporal precision of VS-VTA assemblies displayed a bimodal distribution with peaks around hundred milliseconds and one second. I focused on the more precise temporal scale of VS-VTA assemblies, firstly, not to interfere with the characteristic temporal scales of the experiment such as the stimulus duration and, secondly, because the reward prediction error signals typically involve fast temporal scales in the range of few hundred milliseconds.\\\\
\subsubsection{Task related activity}  
.  To better understand the relation between assembly activations and the task, I analyzed the assembly activation in three task-relevant time windows. The time windows were each of half a second. The first window started directly from stimulus onset to reveal short latency stimulus responses as well as evolving signals with few hundred millisecond delays following stimulus onset. The second window was chose in the half second before reward delivery. These two time windows thus contained most of the activity related to stimulus response and reward anticipation. These time windows had also to be chosen in this way because the time from stimulus onset to reward had been varied in the experiment. Therefore, some sessions have a partial overlap between the two windows, while in others there is a gap between the two windows.  Yet, by choosing the windows in this way, I was able to perform the analyses across sessions. Finally, I used a third window starting from reward delivery on; in unrewarded trials, an expected reward time was chosen, assuming that the animals would have expect the reward after the third lick, since the task was conditioned to animals performing three licks to get the reward. Using this approach, I could show that assemblies composed of different cell types displayed distinct task-related activation patterns.\\
SPN-DAN assembly-pairs were active at few hundred milliseconds after the onset of the stimulus and remained active for several hundred milliseconds. Importantly, SPN-DAN assemblies were preferentially activated by the rewarded stimulus, but not by the non-rewarded. This latter observation suggests that the assemblies could be involved in coding for the predicted value of the stimulus. Of note, only a smaller fraction was active also at the time of the reward retrieval.\\
Conversely, FSN-DAN pairs activated mainly shortly after the onset of the stimulus. Their activation was transient and relatively unselective for the stimulus, meaning that many assemblies were activated both by rewarded and non-rewarded stimuli. In contrast to SPN-DAN assemblies, those among FSN-DAN also activated sharply around reward delivery. These activation patterns suggest that the task-related coding of assemblies containing FSN differs from those comprising SPN in that the assemblies with pallidal contribution code for hedonic rewards and unspecific stimulus salience (relevance), while those involving striatal units serve more specific functions related to reward prediction related to stimulus presentation.\\\\
\subsubsection{Directionality of assembly activity} 
The activation of the underlying units within an assembly can be synchronous and sequential. If sequential, the lag between the two unit activities can reveal directionality. Indeed, dividing the assemblies according to the underlying cell-types revealed that assemblies composed of SPN and DAN neurons were directional with SPN leading DAN. It is worth to note that the directionality was even more consistent when only optogenetically tagged DAN were included compared to inclusion of all DAN characterized by function criteria. Importantly the here observed directionality is functional. Functional directionality means that the units in assemblies do not have to be directly connected by a synapse. Indeed, the SPN-DAN communication may occur both directly and relayed through interspersed neurons (\cite{Ikemoto}). The directionality that was found in SPN-DAN assemblies, but not in FSN-DAN assemblies, suggests that VS could inform VTA about task relevant information. \\\\
\subsubsection{Precision and evolution of assembly activity} 
SPN-DAN assemblies display a high level of variability in their trial-by-trial activation.  It appears important to not here that the sparse firing to stimuli of SPN alone displays a high variability. It is therefore likely that many SPN sum up to drive DAN firing. To further examine the learning dynamics in the assemblies, we modelled the reinforcement learning and examined in how far the assembly activations would encode for RPE.\\\\
\subsubsection{Reinforcement learning model}
In reinforcement learning, Rescorla-Wagner based models implement the main component of the biological reward prediction error signals. This computation is performed as the difference between the actual reward value and the expected reward value, which is adapted trial-by-trial according to error made in the previous trial, thus mimicking the dynamic of the task.\\In this work I modelled a reinforcement learning model (Q L-F) to parameterize learning, on which later I regressed the assembly-pair activity. Q L-F was a hybrid Rescorla-Wagner model with Pearce-Hall update mechanism, which implies a dynamic learning component (\cite{Koppe}, \cite{Li}, \cite{Costa}); and it provided as well a time dependent forgetting component, which accounted for the update of the unchosen option (\cite{ItoDoya1}, \cite{Katahira}). The importance to the account for a a dynamic learning component was highlighted in rats experiment in \citeay{Funamizu}, where the authors compared the inference of fixed learning rates at the beginning and the end of the task. They used 2630 sessions divided in two phases: the first phase contained the first 10 trials and the second phase contained the last 10 trials of the task. The learning rates of the model were independently set to achieve maximum likelihood in each first and last part of a block. Learning rates associated with the first 10 trials were significantly higher than those of the last 10 trials, suggesting that rats utilized time-varying learning rates.\\I assumed that mice utilized a similar learning dynamic in the task proposed; thus I employed a dynamic learning parameter, mimicking the uncertainty of the animal about the possible outcomes.\\In Q L-F the forgetting term was as well a dynamic component. In fact, I assumed that the animals inferred the value related to the unchosen action while taking a specific action.\\Importantly the dynamical forgetting term introduced in Q L-F significantly improved the behavioral fit, suggesting that in the inference of the unchosen option the mice utilized dynamic forgetting parameters.\\Furthemore, Q L-F was compared with other three models not including the dynamic forgetting parameter, and it resulted to be the best model according to likelihood-ratio test (LRT), used the cases in which the models to compare were nested; and to the Bayesian information criterion (BIC), used in cases in which the models to compare had the same number of parameters. Being aware that BIC suffers of some limitations such as a non-selection of relevant parameters for use in model construction, I am planning to implement the Bayesian model comparison, as criterion to compare the models, which automatically includes a penalty for including too much model structure.\\In Q L-F two actions $"$lick$"$/$"$no-lick$"$ and two states CS+/- corresponding to the rewarded/unrewarded stimulus were implemented. In this way, the lick was considered an instrumental action, rather than an impulsive action; the task presented was indeed such that the animal had to learn to lick at the rewarded odor presentation and to not lick at the unrewarded odor; only when both part of the task were well performed the animal  reached the performance criterion. A recent work (\cite{SchultzMot}) showed that in go/no-go tasks the performance reflects balance between impulsive and instrumental components of behavior; in particular behavior is decomposed in two  responses: a primary lick response, defined by a sharp early peak immediately after stimulus onset, which correspond to the impulsive lick, and a broad, multi-peaked secondary response, which corresponds to the instrumental component. I assumed here that the secondary response was driven by the RPE signals. A model that could take into account the shift in behavior from the impulsive to the instrumental component would require a motivational time-component term (\cite{SchultzMot}). Accounting for motivation may have further helped to isolate instrumental components, yet my purpose if this study was to prove  whether the assemblies encoded for RPE components. Therefore a motivation term was not introduced here. However, by modelling a pure Pavlovian model I noticed that relevant results were consistent with Q L-F, although less significant (see \hyperref[chap:SimpRL]{~Appendix \ref*{chap:SimpRL}} for details).\\\\
\subsubsection{Correlation with reinforcement learning model function}
After evaluated which model best approximated the animals$'$ behavior, I focused the further analyses on the crucial terms for RPE signals, namely the uncertainty to get the reward and the prediction error. The prediction error, called $\delta$ in the model, was nothing but the mathematical difference between the expected reward value and the actual reward; the uncertainty, $\alpha$, is a time-dependent component and is modulated by recent predictions and the prediction error $\delta$. $\alpha$ mimicked the uncertainty of the animal to get the reward, and its value was high at the beginning of the task and decreased as the animal learnt the rule, to rise again at the beginning of the reversal phase.\\Conversely, RPE signals in dopamine neurons are such that neuronal activity increases monotonically after the stimulus onset with the probability to get the reward, as consequence of the increase of the certainty of the animal to get the reward when it learnt the task; meanwhile at the reward delivery time, the neural activity decreases as the ability of the animal increases to predict the reward (see \hyperref[chap:Overview]{~Chapter \ref*{chap:Overview}}). In other words, during learning the peak of neural activity shifts back from the reward retrieval time to the stimulus onset.\\Based on this knowledge I assumed that, if an assembly-pair conveyed RPE signals, its activity anti-correlated with the uncertainty in the stimulus window, and correlated with the prediction error in the reward window.\\This was assessed through two regressions of the assembly-pairs activity on the uncertainty about the outcome ($\alpha$) in the CS window and on the prediction error ($\delta$) in the US window. Since the activity of the recorded assemblies followed the Poisson distribution, I employed a generalized linear model with logarithm link function (Poisson regression), which constitutes the generalization of ordinary linear regression for response variables that have error distribution models following the Poisson distribution.\\After proper mathematical transformations (see \hyperref[sec:Regression]{~Chapter \ref*{sec:Regression}}), the regression coefficients of the Poisson regression could be interpreted as the percentage increase of the expected counts (assembly activity) when the covariate variable ($\alpha$ or $\delta$) increases of 1.\\Significance of regression coefficients ($\beta$) was assessed with t-statistic. The t-test performed to assess $\beta$ significance assumes the normality and independence of the regression model residuals. Because of the strong autocorrelation in the data, also given in this case by the learning, regressions on time series often violate this assumption. For this reason, it is good practice to control the residuals autocorrelation before proceeding with the t-test. For our datasets I visually inspected the residuals autocorrelation and confirmed no temporal dependence (probably because fully captured by the model). Nevertheless, as further control I am planning to confirm the significance of the $\beta$ coefficients with a bootstrap test performed with phase randomization (\cite{Mokeichev}). Phase randomization consists in decomposing the data into frequency components, shifting randomly their relative phase, and then summing them back to obtain surrogate time series with power spectrum (and therefore autocorrelation) identical to that of the original data.\\Finally, the  distributions of significant $\beta$ confirmed that SPN-DAN assembly-pairs specifically conveyed RPE signals; these signals were not found in FSN-DAN assembly-pairs, that could instead represent hedonic signals or salience related to the detection of the signal, but not informative about the outcome. In conclusion in this thesis I provided a specific coding mechanism of how assignment value and RPE signals are formed during learning of stimulus-outcome $"$association$"$.\\\\
\subsubsection{Biological meaning}
Altered dopamine signaling is involved in many human disorders, from Parkinson$'$s disease to drug addiction, as well as in the formation of social bonds. Understanding how dopamine signals are generated and encoded may be crucial for a better understanding of physiological behavior and human disorders.\\Toward this aim I studied how the RPE signals are formed in VS-VTA assembly-pairs and further I have shown that different assembly-pairs make specific contribution to RPE dopamine signals.\\\\I found that FSN-DAN interactions were not parameterized by the RPE term formalized in Rescorla-Wagner-like models, and, based on their unselective response, I hypothesized that FSN-DAN pairs were involved in hedonic salience, not informative about the specific outcome. This hypothesis would be compatible with the \citeay{Berridge} motivational salience concept, based on neuronal responses in VP.\\\\Conversely I have shown that SPN-DAN interaction specifically encode the main component of RPE signals, that are informative and predictive about the outcome. RPE signals are thought to be the basis of the learning and thus reflect high-level behavior, thus these findings may help to understand dopamine actions both in physiological behaviors and psychiatric disorders. Importantly studies highlighted that numerous behaviors, including choice, confidence, contextual expectations, can be modulated on the basis of  RPE computations (\cite{Gadagkar}, \cite{Stauffer}). Further learnt natural behaviors like the formation of social bonds are presumably related on VS-VTA interactions (\cite{Ungless2004}, \cite{Walum}).\\\\These finding suggested that interactions between VS-VTA convey not-outcome specific signals and outcome selective and predictive signals through different functional pathways. These encoding features together guarantee the full prediction coding.


