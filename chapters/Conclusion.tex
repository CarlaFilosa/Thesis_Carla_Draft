\chapter{Conclusion}
\label{chap:Conclusion}
Subjects learn to assign value to stimuli that predict outcomes. Novelty, rewards or punishment evoke reinforcing phasic dopamine release from midbrain neurons to ventral striatum that mediates expected value and salience of stimuli in humans and animals. It is however not clear whether phasic dopamine release is sufficient to generate distinct engrams that encode salient stimuli in ventral striatum that may inform dopaminergic neurons to respond with a prediction signal. We addressed this question in awake mice. Evoked phasic dopamine induced plasticity selectively to the population encoding of coincidently presented stimuli and increased their distinctness from other stimuli. Phasic dopamine thereby enhanced the decoding of previously paired stimuli and increased their perceived salience. 
During reinforcement learning such dynamics progressively generated functional directional assemblies between striatal projection neurons and dopaminergic neurons. These findings provide a network coding mechanism of how dopaminergic learning signals promote value assignment to generate an assembly prediction code to dopaminergic midbrain neurons.
