\section*{Summary}
Reward is an object, stimulus, situation or activity that evokes positive emotions such as pleasure and desire. Thus reward induces approach behavior and is maximised by an agent in decision making during learning. Maximisation of the reward during a learning process requires prediction, namely information about the future. An error can be defined in the most general sense as a discrepancy between what is happening and what is predicted to happen. A reward prediction error, then, is the difference between a reward that is being received and the reward that is predicted to be received. Reward prediction error signals are essential to learning and they are supposed to be encoded by dopamine neurons through dopamine pathways. In particular the mesolimbic dopamine pathway comprising the ventral tegmental area (VTA) and projection terminals in the ventral striatum (VS) has been identified as a critical neural system involved in processing reward prediction error signals. In this thesis we proposed a study on formation of reward prediction error signals in VS-VTA interregional assemblies during learning. We recorded in-vivo electrophysiological data in VS, including ventral pallidum (VP), and VTA during a reversal go/no go task in mice. VS/VP units were classified in striatal projection units (SPN), fast spiking neurons (FSN) and cholinergic interneurons (CIN), whereas VTA units were classified in dopamine neurons (DAN), gabaergic neurons (GABA) and glutammatergic neurons (GLU).
